\begin{figure}[ht]
	\centering
	\begin{tikzpicture}
		\begin{feynman}
			% disconnected t-channel diagram
			\vertex (a1) at (0,1.5);
			\vertex (a2) at (0,-1.5);
			\vertex (b1) at (2,1.5);
			\vertex (b2) at (2,-1.5);
			\vertex (c) at (1, 0);

			\diagram*{
				(a1) -- [plain] (c) -- [plain] (b1),
				(a2) -- [plain] (c) -- [plain] (b2),
			};

			\vertex (f1) at (2.25, 0);
			\vertex (fup) at (2.25, 0.75);
			\vertex (fdown) at (2.25, -0.75);

			\diagram*{
				(f1) -- [plain, half left] (fup),
				(f1) -- [plain, half right] (fup),
				(f1) -- [plain, half left] (fdown),
				(f1) -- [plain, half right] (fdown),
			};

		\end{feynman}
	\end{tikzpicture}
	\hspace{3cm}
	\begin{tikzpicture}
		\begin{feynman}
			% disconnected t-channel diagram
			\vertex (a1) at (0,1.5);
			\vertex (a2) at (0,-1.5);
			\vertex (b1) at (2,1.5);
			\vertex (b2) at (2,-1.5);
			\vertex (c) at (1, 0);

			\diagram*{
				(a1) -- [plain] (c) -- [plain] (b1),
				(a2) -- [plain] (c) -- [plain] (b2),
			};

			\vertex (loopv1) at (1.33, 0.5);  % Place the loop vertex near the upper leg
			\vertex (loopv2) at (1.67, 1);  % Place the loop vertex near the upper leg

			\diagram*{
				(loopv1) -- [plain, half right, min distance=5mm] (loopv2),
			};
		\end{feynman}
	\end{tikzpicture}
	\vspace{5mm}
	\caption{Examples of a disconnected (left) and an un-amputated (right) Feynman diagram.}
	\label{fig:01_qft_interactions_feynman_disconnected}
\end{figure}