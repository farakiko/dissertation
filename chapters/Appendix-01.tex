\chapter{Supplementary Material for Chapter~\ref{sec:01_symmetries}}
\label{app:01}

\section{Symmetries in physics}
\label{app:01_ymmetries}

\subsection{Derivation of the Poincaré algebra}
\label{app:01_poincare_algebra}

Perhaps the simplest way to derive these is via the infinite-dimensional representation of the generators as differential operators acting on functions of spacetime $\psi(x^\mu)$:
\begin{equation}
	\label{eq:01_poincare_generators_diff_representation}
	\begin{split}
		P_\mu &= -i\partial_\mu, \\
		M_{\mu\nu} &= i(x_\mu \partial_\nu - x_\nu \partial_\mu).
	\end{split}
\end{equation}
These should be familiar as the momentum and angular momentum operators from classical and quantum mechanics, generalized to include boosts and the time dimension.
Thus, for example,
\begin{equation}
	\label{eq:01_poincare_commutation_derivation_example}
	\begin{split}
	[M_{0i}, P_j]\psi(x^\mu) 
		&= (-i^2)[x_0\partial_i - x_i\partial_0, \partial_j]\psi \\
		&= [(x_0\partial_i\partial_j - x_i\partial_0\partial_j) - (\partial_j(x_0\partial_i) - \partial_j(x_i\partial_0))]\psi \\
		&= [(x_0\partial_i\partial_j - x_i\partial_0\partial_j) - (x_0\partial_i\partial_j - \eta_{ij}\partial_0 -  x_i\partial_0\partial_j)]\psi \\
		&= \eta_{ij}\partial_0 \psi \\
		&= i\eta_{ij}P_0\psi.
	\end{split}
\end{equation}
The rest of the commutation relations in Eqs.~\ref{eq:01_poincare_algebra} and~\ref{eq:01_poincare_algebra_mmunu} can be derived similarly.


\chapter{Supplementary Material for Chapter~\ref{sec:01_qft}}
\label{app:01_qft}

\section{Classical field theory}
\label{app:01_qft_classical}

\subsection{Lagrangian mechanics}
\label{sec:01_qft_classical_lagrangian}

Lagrangian mechanics is a formulation of classical mechanics based on the energies of a system, as opposed to the force-based Newtonian approach.
We define the Lagrangian of a particle as the difference between its kinetic ($T$) and potential energies ($V$):
\begin{equation}
	\label{eq:01_qft_classical_lagrangian}
	L(\dot x, x) = T(\dot x) - V(x),
\end{equation}
where $x$ and $\dot x$ are the particle's position and velocity, respectively.
To determine the dynamics of the system, we assign a value based on $L$ to each possible path the particle can take between two points $t_i$ and $t_f$, called the \textit{action} $S$:
\begin{equation}
	\label{eq:01_qft_classical_action}
	S[x(t)] = \int_{t_i}^{t_f} L(\dot x(t), x(t)) dt.
\end{equation}
The equations of motion (EOMs) are then derived from the \textit{principle of stationary action}, which states that the true path is an extremum of $S$.
This condition yields the Euler-Lagrange (E-L) equations:
\begin{equation}
	\label{eq:01_qft_classical_euler_lagrange}
	\frac{d}{dt}\left(\frac{\partial L}{\partial \dot x}\right) - \frac{\partial L}{\partial x} = 0.
\end{equation}

\begin{example}
\label{ex:01_qft_lagrangian_classical_newton}
We can confirm that this is equivalent to Newtonian mechanics by considering the simple Lagrangian:
\begin{equation}
\label{eq:01_qft_classical_lagrangian_newton}
L = \frac{1}{2}m\dot x^2 - V(x).
\end{equation}
Plugging this into Eq.~\ref{eq:01_qft_classical_euler_lagrange} gives us:
\begin{equation}
	\label{eq:01_qft_classical_euler_lagrange_newton}
	m\ddot x + \frac{d V}{d x} = 0 \quad\Rightarrow m\ddot x = -\frac{d V}{d x} = F,
\end{equation}
which is exactly Newton's second law.
Classically, Lagrangian mechanics has certain benefits over Newtonian mechanics, such as being based on scalars (energies) instead of vectors (forces), and ease of coordinate transformations.
For us, as we will see, its main advantage is its natural generalization to fields rather than particles.
\end{example}

\subsubsection{Path integral formulation of QM}
\label{sec:01_qft_classical_path_integral}

Note that the principle of stationary action is based on the classical behavior of particles, in that they follow a single true path.
However, in QM, (unobserved) particles are thought to traverse a superposition of all possible paths between two observed positions.
This can be expressed with Feynman's path integral formula, where the probability of observing a particle at position $q_f$ and time $T$ given it was at $q_i$ at $t = 0$ is based on its wavefunction
\begin{equation}
	\label{eq:01_qft_lagrangian_path_integral}
	\psi(q_f, T) = \int_{q_i}^{q_f} \mathcal{D}q(t) e^{iS[q(t)]/\hslash},
\end{equation}
where $\int_{q_i}^{q_f}\mathcal{D}q(t)$ is an integral over all possible paths $q(t)$ between $q_i$ and $q_f$, interfering through their complex phases $e^{iS[q(t)]/\hslash}$ that are based on the action $S[q(t)]$ of the path divided by the reduced Planck constant $\hslash$.
In the classical limit $\cnicefrac{\hslash}{S} \rightarrow 0$,\footnote{If we take $S \sim \mathrm{energy} \cdot T = \frac{\hslash c}{\lambda}\cdot T$, then the classical limit $\frac{\hslash}{S} = \frac{\lambda}{cT} \rightarrow 0$ physically is the case where the de Broglie wavelength of the particle $\lambda$ is negligible compared to the relevant length scales.} by the stationary phase approximation, only the path that extremizes the action contributes, as we expect.

The path integral formulation was a critical development in QFT.
The fact that the Lagrangian shows up naturally in this formulation is the reason why we ``consider it the most fundamental specification of a QFT'' (Peskin and Schroeder~\cite{Peskin:1995ev} Chapter 9).


\subsection{Solutions to the Klein-Gordon equation}

Using the Fourier transform, we see the solutions to the Klein-Gordon equation are plane waves:
\begin{equation}
	\label{eq:01_qft_field_kg_solutions}
	\phi(\cvec x, t) = \int \frac{d^3p}{(2\pi)^3} \tilde \phi(\cvec{p}, t) e^{i\cvec{p}\cdot\cvec{x}},
\end{equation}
with $\tilde \phi(\cvec{p}, t)$ satisfying the simple-harmonic oscillator (SHO) equation
\begin{equation}
	\label{eq:01_qft_field_kg_sho}
	(\partial_t^2 - |\cvec{p}|^2 - m^2)\tilde \phi(\cvec{p}, t) = 0 \quad \Rightarrow \quad \tilde \phi(\cvec{p}, t) \propto e^{-i\omega_p t},
\end{equation}
with frequency $\omega_p = \abs{\sqrt{|\cvec{p}|^2 + m^2}}$.
Thus, 
\begin{equation}
	\label{eq:01_qft_field_kg_solutions_final}
	\phi(\cvec x, t) = \int \frac{d^3p}{(2\pi)^3} \frac{1}{\sqrt{2\omega_p}} (a(\cvec{p})e^{ip\cdot x} + a^*(\cvec{p})e^{-ip\cdot x}),
\end{equation}
where $p\cdot x$ is the 4D spacetime inner product with $p_\mu = (\omega_p, \cvec{p})$, and the $\cnicefrac{1}{\sqrt{2\omega_p}}$ factor is conventional.
The coefficients $a$ and $a^*$ are complex conjugates to ensure a real sum.
As we will see, in \textit{quantum} field theory, the form of the fields is quite similar but with $a$ and $a^*$ quantum operators.


\subsection{Hamiltonian mechanics}

In QM, the Hamiltonian formalism is most natural.
In QFT, as well, it will prove useful for the \textit{canonical quantization} of the fields in the next section.
The Hamiltonian density is the Legendre transform of the Lagrangian:
\begin{equation}
	\label{eq:01_qft_hamiltonian}
	\mathcal H = \pi_a\dot\phi_a - \mathcal{L},
\end{equation}
where $\dot\phi$ is the time derivative and
\begin{equation}
	\label{eq:01_qft_hamiltonian_momenta}
	\pi_a = \frac{\partial\mathcal L}{\partial\dot\phi_a}
\end{equation}
are the \textit{conjugate momenta} to the fields $\phi_a$.
The Hamiltonian generally has the interpretation of the energy of a system, or the energy operator in QM.
The EOMs are Hamilton's equations:
\begin{equation}
	\label{eq:01_qft_hamiltonian_eoms}
	\begin{split}
		\dot\phi_a &= \frac{\partial\mathcal H}{\partial\pi_a}, \\
		\dot\pi_a &= -\frac{\partial\mathcal H}{\partial\phi_a}.
	\end{split}
\end{equation}

\subsubsection{Poisson brackets}

The time evolution of a general quantity $f(\phi, \pi)$ can be expressed as:
\begin{equation}
	\label{eq:01_qft_hamiltonian_time_evolution_f}
	\frac{df(\phi, \pi)}{dt} = \frac{\partial f}{\partial\phi}\dot\phi + \frac{\partial f}{\partial\pi}\dot\pi = \frac{\partial f}{\partial\phi}\frac{\partial\mathcal H}{\partial\pi} - \frac{\partial f}{\partial\pi}\frac{\partial\mathcal H}{\partial\phi} \equiv \{f, \mathcal H\},
\end{equation}
where the last step defines the \textit{Poisson bracket} $\{\cdot, \cdot\}$.
In terms of Poisson brackets, Hamilton's equations can be written as:
\begin{equation}
	\label{eq:01_qft_hamiltonian_eoms_poisson}
	\begin{split}
		\dot\phi_a &= \{\phi_a, \mathcal H\} = \frac{\partial\mathcal H}{\partial\pi_a}, \\
		\dot\pi_a &= \{\pi_a, \mathcal H\} = -\frac{\partial\mathcal H}{\partial\phi_a}.
	\end{split}
\end{equation}
Importantly, the \textit{canonical} fields of the Hamiltonian, $\phi$ and $\pi$, obey the canonical Poisson bracket relations:
\begin{equation}
	\label{eq:01_qft_hamiltonian_canonical_poisson}
	\begin{split}
		\{\phi(\cvec{x}), \phi(\cvec{y})\} &= 0, \\
		\{\pi(\cvec{x}), \pi(\cvec{y})\} &= 0, \\
		\{\phi(\cvec{x}), \pi(\cvec{y})\} &= \delta^3(\cvec{x} - \cvec{y}).
	\end{split}
\end{equation}

\begin{example}
\label{ex:01_qft_hamiltonian}
Revisiting the simple (non-field-theoretic) Lagrangian from Example~\ref{ex:01_qft_lagrangian_classical_newton}, we can derive the conjugate momentum to $x$ to be:
\begin{equation}
	\label{eq:01_qft_hamiltonian_example_momenta}
	p = \frac{\partial L}{\partial\dot x} = m\dot x,
\end{equation}
and hence,
\begin{equation}
	\label{eq:01_qft_hamiltonian_example}
	H = p\dot x - L = \frac{1}{2}m\dot x^2 + V(x) = \frac{p^2}{2m} + V(x),
\end{equation}
which is the classical energy of a free particle.
Note, as in the last step, we express the Hamiltonian as a function of the conjugate momenta $p$ rather than the time derivative of the coordinate $\dot x$.
Finally, the EOMs are:
\begin{equation}
	\label{eq:01_qft_hamiltonian_example_eoms}
	\begin{split}
		\dot x &= \frac{\partial H}{\partial p} = \frac{p}{m}, \\
		\dot p &= -\frac{\partial H}{\partial x} = -\frac{dV}{dx}.
	\end{split}
\end{equation}
The former is simply the definition of velocity, while the latter again reproduces Newton's second law.
Finally, we can explitly confirm the canonical Poisson bracket relations for the canonical coordinates $x$ and $p$:
\begin{equation}
	\label{eq:01_qft_hamiltonian_example_poisson}
	\begin{split}
		\{x, x\} &= \{p, p\} = 0, \\
		\{x, p\} &= \frac{\partial x}{\partial x}\frac{\partial p}{\partial p} - \frac{\partial x}{\partial p}\frac{\partial p}{\partial x} = 1.
	\end{split}
\end{equation}
\end{example}

\subsubsection{Free scalar field Hamiltonian}

For the free scalar Lagrangian in Eq.~\ref{eq:01_qft_field_kg_lagrangian}, we find
\begin{equation}
	\label{eq:01_qft_hamiltonian_kg_momenta}
    \pi = \dot\phi = -i \int \frac{d^3p}{(2\pi)^3} \sqrt{\frac{\omega_p}{2}} (a(\cvec{p})e^{ip\cdot x} - a^*(\cvec{p})e^{-ip\cdot x}),
\end{equation}
where we plugged in the plane-wave solutions for $\phi$ from Eq.~\ref{eq:01_qft_field_kg_solutions_final}, and
\begin{equation}
    \label{eq:01_qft_hamiltonian_kg}
    \mathcal H = \pi\dot\phi - \mathcal L = \frac{1}{2}\pi^2 + \frac{1}{2}(\nabla\phi)^2 + \frac{1}{2}m^2\phi^2.
\end{equation}
This is, in fact, the same as the expression for energy we derived via Noether's theorem in Eq.~\ref{eq:01_qft_symmetries_charge_translation_kg}.
Note that, unlike the Lagrangian, the Hamiltonian is not Lorentz-invariant.
This makes sense under the interpretation of the Hamiltonian as the energy, which is not a Lorentz scalar.
Its Lorentz-invariance, as well as its natural connection to the path integral formulation (Section~\ref{sec:01_qft_classical_path_integral}), is the reason the Lagrangian viewpoint is preferred in QFT.



\section{Quantization}
\label{app:01_qft_quantization}

In this section, we briefly sketch \textit{canonical quantization}, a process of turning a classical field theory into a QFT.
It is based on the Hamiltonian formalism, in close analogy to the quantization of classical mechanics $\rightarrow$ QM.
The result makes manifest the connection between quantum fields and their associated particles.

An alternative quantization approach not discussed here is based on the path integral formulation (see Section~\ref{sec:01_qft_classical_path_integral}). 
As with most alternative mathematical prescriptions of the same physics, it provides useful insight into the theory and can simplify certain calculations.
Further detail can be found, for example, in Peskin and Schroeder~\cite{Peskin:1995ev} Chapter 9.

\subsection{Canonical quantization}

The process of quantizing a classical system in QM can be summarized as (1) promoting the canonical coordinates to quantum operators, and (2) imposing the canonical Poisson bracket relations as quantum commutator relations:
\begin{equation}
	\label{eq:01_qft_quantization}
	\begin{split}
		x \rightarrow \hat x,& \quad p \rightarrow \hat p, \\
		\{x, p\} = 1 &\rightarrow [\hat x, \hat p] = i\hslash.
	\end{split}
\end{equation}
Canonical quantization of a field theory is done analogously, with fields becoming operator-valued and obeying their own canonical commutation relations based on Eq.~\ref{eq:01_qft_hamiltonian_canonical_poisson}.

For our free scalar field theory (Eq.~\ref{eq:01_qft_field_kg_lagrangian}), this means promoting the integration constants in the classical solution (Eq.~\ref{eq:01_qft_field_kg_solutions_final} and~\ref{eq:01_qft_hamiltonian_kg_momenta}) to operators:
\begin{equation}
	\label{eq:01_qft_quantization_fsf_fields}
    \begin{split}
    \phi(\cvec x, t) &= \int \frac{d^3p}{(2\pi)^3} \frac{1}{\sqrt{2\omega_p}} (\hat a_{\cvec{p}}\,e^{ip\cdot x} + \hat a^\dagger_{\cvec{p}}\, e^{-ip\cdot x}), \\
    \pi(\cvec x, t) &= -i\int \frac{d^3p}{(2\pi)^3} \sqrt{\frac{\omega_p}{2}} (\hat a_{\cvec{p}}\,e^{ip\cdot x} - \hat a^\dagger_{\cvec{p}}\, e^{-ip\cdot x}),
    \end{split}
\end{equation}
where again $p \cdot x = p_\mu x^\mu$ is the 4D spacetime inner product and $p_\mu = (\omega_p = \sqrt{\abs{\cvec{p}}^2 + m^2}, \cvec{p})$.
Recall that the integration constants $a(\cvec{p})$ and $a^*(\cvec{p})$ arose from a SHO equation for each momentum $\cvec{p}\,$ (Eq.~\ref{eq:01_qft_field_kg_sho}); thus, quantized, we expect them to correspond to the raising ($\hat a^\dagger$) and lowering ($\hat a$) operators of a quantum harmonic oscillator (QHO), again one for each momentum mode $\cvec{p}$.

We can check this by deriving their commutation relations.
%  and with energy levels $\omega_p = \abs{\sqrt{\cvec{p}^2 + m^2}}$.
Indeed, imposing the canonical commutation relationships:
\begin{equation}
	\label{eq:01_qft_quantization_fsf_commutators}
	[\phi(\cvec{x}, t), \phi(\cvec{y}, t)] = [\pi(\cvec{x}, t), \pi(\cvec{y}, t)] = 0, \quad [\phi(\cvec{x}, t), \pi(\cvec{y}, t)] = i\delta^3(\cvec{x} - \cvec{y}),
\end{equation}
reproduces (continuous versions of) the raising and lowering operator commutation relationships for a QHO:
\begin{equation}
	\label{eq:01_qft_quantization_fsf_commutators_qho}
	[\hat a_{\cvec{p}}, \hat a_{\cvec{q}}] = [\hat a^\dagger_{\cvec{p}}, \hat a^\dagger_{\cvec{q}}] = 0, \quad [\hat a_{\cvec{p}}, \hat a^\dagger_{\cvec{q}}] = (2\pi)^3\delta^3(\cvec{p} - \cvec{q}).
\end{equation}
Next, we look at the commutators with the Hamiltonian and the resulting Hilbert space.

\subsection{The Hamiltonian and the vacuum catastrophe}
\label{sec:01_qft_quantization_hamiltonian}

The quantized Hamiltonian, from Eq.~\ref{eq:01_qft_hamiltonian_kg}, can be found to be:
\begin{equation}
	\label{eq:01_qft_quantization_hamiltonian_zeropoint}
	\begin{split}
		H &= \int d^3x\ \left(\frac{1}{2}\pi^2 + \frac{1}{2}(\nabla\phi)^2 + \frac{1}{2}m^2\phi^2\right) \\
		% &= \int \frac{d^3p}{(2\pi)^3} \frac{1}{2\omega_p} (\hat a_{\cvec{p}}\hat a^\dagger_{\cvec{p}} + \hat a^\dagger_{\cvec{p}}\hat a_{\cvec{p}}) + \frac{\omega_p}{2}(\hat a_{\cvec{p}}\hat a^\dagger_{\cvec{p}} + \hat a^\dagger_{\cvec{p}}\hat a_{\cvec{p}}) \\
		&= \int \frac{d^3p}{(2\pi)^3} \omega_p [\hat a^\dagger_{\cvec{p}}\,\hat a_{\cvec{p}} + \frac{1}{2} (2\pi)^3\delta^3(0)].
	\end{split}
\end{equation}
This looks a lot like the Hamiltonian for a QHO, $H = \omega(a^\dagger a + \cnicefrac{1}{2})$, for each momenta, but with an unwieldy delta function.
This latter term is called the \textit{zero-point energy} and represents the energy of the vacuum state.
It is infinite, and, indeed, is one of the many infinities that have to be dealt with in QFT.

In this case, since it is a constant energy term, it does not affect the dynamics of the system and can simply be ignored / subtracted for our purposes.\footnote{Equivalently, we can consider the \textit{normal-ordered} Hamiltonian (see, e.g., Tong QFT~\cite{TongQFT} Chapter 2.3).
An alternative way of resolving the infinity is to introduce an ultra-violet cut-off scale $\Lambda$ in the integral over momenta.}
However, the vacuum energy density \textit{does} affect Einstein's equations of general relativity, and the disagreement between the large zero-point energy we expect from QFT and the small observed value is known as the \textit{cosmological constant problem} (or, more dramatically, the \textit{vacuum catastrophe})~\cite{Adler:1995vd, Bengochea:2019daa}.

\subsubsection{An infinity of harmonic oscillators}

Subtracting away the zero-point energy gives us
\begin{equation}
    \label{eq:01_qft_quantization_hamiltonian}
    H = \int \frac{d^3p}{(2\pi)^3} \omega_p \hat a^\dagger_{\cvec{p}}\,\hat a_{\cvec{p}},
\end{equation}
whose commutators with the raising and lowering operators are:
\begin{equation}
    \label{eq:01_qft_quantization_hamiltonian_commutators}
    [H, \ophatd a] = \svecp \omega \ophatd a, \quad [H, \ophat a] = -\svecp \omega \ophat a, 
\end{equation}
just as for a QHO.
This tells us that given an eigenstate of $H$, $\ket{E}$, with eigenvalue $E$, $\ophatd a \ket{E}$ and $\ophat a \ket{E}$ are also eigenstates with eigenvalues $E + \svecp \omega$ and $E - \svecp \omega$, respectively:
\begin{equation}
    \label{eq:01_qft_quantization_hamiltonian_eigenstates}
    \begin{split}
        H\ophatd a \ket{E} &= (\ophatd a H + \svecp \omega \ophatd a) \ket{E} = (E + \svecp \omega) \ophatd a \ket{E}, \\
        H\ophat a \ket{E} &= (\ophat a H - \svecp \omega \ophat a) \ket{E} = (E - \svecp \omega) \ophat a \ket{E}.
    \end{split}
\end{equation}

% To derive the Hilbert space of this theory, we first define the vacuum $\ket{0}$ as the state for which 
% \begin{equation}
%     \label{eq:01_qft_quantization_hamiltonian_vacuum}
%     \ophat a \ket{0} = 0\; \forall \cvec{p} \quad \Rightarrow \quad H\ket{0} = 0.
% \end{equation}
% The remaining eigenstates are then obtained by (repeatedly) acting with $\ophatd a$ on the vacuum:
% \begin{equation}
%     \label{eq:01_qft_quantization_hamiltonian_states}
%     \begin{split}
%         \ket{\cvec{p}} \propto \ophatd a\ket{0} \quad &\Rightarrow \quad H\ket{p} = \svecp \omega\ket{p}. \\
%         \ket{\cvec{p}_1, \cvec{p}_2} \propto \hat a^\dagger_{\cvec{p}_1}\hat a^\dagger_{\cvec{p}_2}\ket{0} \quad &\Rightarrow \quad H\ket{\cvec{p}_1, \cvec{p}_2} = (\omega_{\cvec{p}_1} + \omega_{\cvec{p}_2})\ket{\cvec{p}_1, \cvec{p}_2} \\
%         &\vdots \\
%         \ket{\cvec{p}_1, \ldots, \cvec{p}_n} \propto \hat a^\dagger_{\cvec{p}_1} \ldots\hat a^\dagger_{\cvec{p}_n}\ket{0} \quad &\Rightarrow \quad H\ket{\cvec{p}_1, \ldots, \cvec{p}_n} = \bigg(\sum_{i=1}^n \omega_{\cvec{p}_i}\bigg) \ket{\cvec{p}_1,  \ldots, \cvec{p}_n}.
%     \end{split}
% \end{equation}
% This is essentially the sum of the Hilbert spaces of an infinite number of QHOs, across all momenta.
% It is called the \textit{Fock space}.

\subsection{Particles}
\label{sec:01_qft_quantization_particles}

To understand these states further, we can also quantize the total momentum of the field density we found from Noether's theorem (Eq.~\ref{eq:01_qft_symmetries_charge_translation_kg}):\footnote{Technically, we show here the \textit{normal-ordered} momentum.}
\begin{equation}
    \label{eq:01_qft_quantization_momentum}
    \cvec{P} = \int d^3x\ \pi\cvec{\nabla}\phi = \int \frac{d^3p}{(2\pi)^3} \cvec{p}\, \ophatd a \ophat a.
\end{equation}
Acting with $\cvec{P}$ on $\ket{\cvec{p}}$ gives us:
\begin{equation}
    \label{eq:01_qft_quantization_momentum_states}
    \begin{split}\,
    \cvec{P}\ket{\cvec{p}} &= \int \frac{d^3k}{(2\pi)^3} \cvec{k}\, \hat a^\dagger_{\cvec{k}}\, \hat a_{\cvec{k}} \hat a^\dagger_{\cvec{p}} \ket{0} \\
    &= \int \frac{d^3k}{(2\pi)^3} \cvec{k}\, \hat a^\dagger_{\cvec{k}}\, \big(\ophatd a \hat a_{\cvec{k}} - (2\pi)^3 \delta^3(\cvec{k} - \cvec{p})\big) \ket{0} \\
    &= \cvec{p}\ket{\cvec{p}}.
    \end{split}
\end{equation}
Thus the states $\ket{\cvec{p}}$ are eigenstates of $\cvec{P}$ as well, with eigenvalues $\cvec{p}$.
Putting this together, we have a Hilbert space spanned by the states $\ket{\cvec{p}}$, which each have momentum $\cvec{p}$ and energy $\omega_{\cvec{p}} = \abs{\sqrt{\cvec{p}^2 + m^2}}$, i.e. the relativistic energy-momentum relation for a free particle.

Thus, we see $\ket{\cvec{p}}$ exactly corresponds to the momentum eigenstate for a single particle of mass $m$ and momentum $\cvec{p}$!
One can similarly quantize the total angular momentum of the field $\cvec{J}$ and show that $\cvec{J}\ket{\cvec{p} = 0} = 0$, i.e. the particle has spin 0.

This is one of the miracles of QFT: what from quantizing a free, relativistic field looked bizarrely like an infinite series of QHOs, actually gives the intuitive physical result of discrete particle states.
The Fock space hence is the space spanned by different numbers of discrete particles per each continuous momentum mode $\cvec{p}$.
The number of particles $n$ in a particular state of the Fock space is given by the number operator $N$, essentially the Hamiltonian density divided by $\svecp \omega$:
\begin{equation}
    \label{eq:01_qft_quantization_number}
    N = \int \frac{d^3p}{(2\pi)^3} \ophatd a \ophat a \quad \Rightarrow \quad N\ket{\cvec{p}_1, \ldots, \cvec{p}_n} = n\ket{\cvec{p}_1, \ldots, \cvec{p}_n}.
\end{equation}
Note that the number operator $N$ commutes with the Hamiltonian $H$, $[N, H] = 0$, which means particle number is conserved; however, this will not be the case for interacting theories in the next section.

% Finally, observe that for our scalar field, the creation operators commute amongst themselves. 
% This means the states $\ket{\cvec{p}_1, \ldots, \cvec{p}_n}$ are symmetric under exchange of particles, and thus describe \textit{bosons}.

\subsubsection{Normalization of states and wavepackets}

% Note that we have not chosen a normalized the momentum eigenstates in Eq.~\ref{eq:01_qft_quantization_hamiltonian_eigenstates}.
Note that we cannot simply choose the normalization of momentum eigenstates as $\braket{\cvec{q}|\cvec{p}} = \delta^3(\cvec{q} - \cvec{p})$, as in nonrelativistic QM, because the delta function alone is not Lorentz-invariant.
Instead, we choose the normalization in Eq.~\ref{eq:01_qft_quantization_normalization}, which is a Lorentz scalar.
% \begin{equation}
% 	\label{eq:01_qft_quantization_normalization}
% 	\ket{\cvec{p}} = \sqrt{2\svecp E}\, \ophatd a \ket{0} \quad \Rightarrow \quad \braket{\cvec{q}|\cvec{p}} = 2\svecp E\, \delta^3(\cvec{q} - \cvec{p}),
% \end{equation}
% which is a Lorentz scalar.

% \subparagraph{Normalization of states and wavepackets} We choose to normalize the momentum eigenstates as $\ket{\cvec{p}} = \sqrt{2\svecp E}\, \ophatd a \ket{0}$.
% This is chosen carefully so that the inner product between two states $\braket{\cvec{q}|\cvec{p}} = 2\svecp E\, \delta^3(\cvec{q} - \cvec{p})$ is Lorentz-invariant.
Like in QM, however, these momentum eigenstates are not normalized to 1: $\braket{\cvec{p}|\cvec{p}} = 2\svecp E\delta^3(0)$, so they are not exactly physical one-particle states.
Physical particles must exist in the form of a wavepacket:
\begin{equation}
	\label{eq:01_qft_quantization_wavepacket}
	\ket{\varphi} = \int d^3p\, \varphi(\cvec{p})\ket{\cvec{p}},
\end{equation}
with some spread in momenta $\varphi(\cvec{p})$.
However, as long as this variation is smaller than the resolution of our detector (as we will assume), for all practical purposes and calculations we can continue to treat particles as momentum eigenstates.
This assumption is further motivated in Peskin and Schroeder~\cite{Peskin:1995ev} Chapter 4.5.

\subsection{The complex scalar field and antiparticles}
\label{sec:01_qft_quantization_complex}

The complex scalar field Lagrangian from Eq.~\ref{eq:01_qft_symmetries_complex_lagrangian} has the EOMs:
\begin{equation}
    \label{eq:01_qft_quantization_complex_eoms}
    \begin{split}
        (\partial_\mu\partial^\mu + m^2)\psi &= 0, \\
        (\partial_\mu\partial^\mu + m^2)\psi^* &= 0,
    \end{split}
\end{equation}
with solutions:
\begin{equation}
    \label{eq:01_qft_quantization_complex_solutions}
    \begin{split}
        \psi(x) &= \int \frac{d^3p}{(2\pi)^3} \frac{1}{\sqrt{2\omega_p}} (b(\cvec{p})e^{ip\cdot x} + c^*(\cvec{p}) e^{-ip\cdot x}).
        % \psi^*(x) &= \int \frac{d^3p}{(2\pi)^3} \frac{1}{\sqrt{2\omega_p}} (a^*(\cvec{p})\,e^{ip\cdot x} + b(\cvec{p}) e^{-ip\cdot x}).
    \end{split}
\end{equation}
Note that because the field is complex, the coefficients $b$ and $c^*$ need not be complex conjugates of each other as for a real field.
This field can be quantized analogously to above:
\begin{equation}
    \label{eq:01_qft_quantization_complex_fields}
    \begin{split}
        \psi(\cvec x, t) &= \int \frac{d^3p}{(2\pi)^3} \frac{1}{\sqrt{2\omega_p}} (\hat b_{\cvec{p}}\,e^{ip\cdot x} + \hat c^\dagger_{\cvec{p}}\, e^{-ip\cdot x}), \\
        \psi^\dagger(\cvec x, t) &= \int \frac{d^3p}{(2\pi)^3} \frac{1}{\sqrt{2\omega_p}} (\hat b^\dagger_{\cvec{p}}\,e^{-ip\cdot x} + \hat c_{\cvec{p}}\, e^{ip\cdot x}),
    \end{split}
\end{equation}
where we now have two sets of creation and annihilation operators, $\{\hat b^\dagger, \hat b\}$ and $\{\hat c^\dagger, \hat c\}$.
One can check each pair individually satisfies the canonical commutation relations from Eq.~\ref{eq:01_qft_quantization_fsf_commutators_qho}, and mutually commutes with each other.

Thus, they are interpreted as corresponding to two different particles, with the same mass $m$ and spin 0, \textit{but}, as we saw, with opposite charges under the \UU[1] internal symmetry.
Such pairs are considered particles and antiparticles.

Finally, let us revisit and quantize the conserved charge associated with the \UU[1] symmetry (Eq.~\ref{eq:01_qft_symmetries_u1_current_charge}):
\begin{equation}
    \label{eq:01_qft_quantization_complex_charge}
    Q = \int d^3x\ i(\psi^*\partial^0\psi - \psi\partial^0\psi^*) \rightarrow \int \frac{d^3p}{(2\pi)^3} \left(\hat b^\dagger_{\cvec{p}}\,\hat b_{\cvec{p}} - \hat c^\dagger_{\cvec{p}}\,\hat c_{\cvec{p}}\right) = N_b - N_c.
\end{equation}
This is saying the difference in the number of particles and antiparticles is conserved, which for a single charged particle-antiparticle pair, is equivalent to charge conservation.
This will be more significant for interacting theories, in which $N_b$ and $N_c$ are not individually conserved but as long as the interactions retain the \UU[1] symmetry, $Q$ is.

\subsubsection{Negative energy states?} 

Note that the full, time-dependent formula for the field $\psi(\cvec{x}, t)$ contains both the $e^{-ip\cdot x}\propto e^{-iEt}$ and $e^{ip\cdot x}\propto e^{iEt}$ terms.
As single-particle plane-wave solutions to the nonrelativistic Schr\"odinger equation, these would correspond to positive and negative energy states, the latter of which does not make physical sense.\footnote{This is related to the problem Dirac faced in developing his relativistic quantum theory of the electron, except we are dealing with bosons instead of fermions, so we cannot rely on the fermionic Dirac sea ``solution''.}
Our solution is to refer to these states instead as \textit{positive-} and \textit{negative-frequency} modes, which, as we saw, are always associated to operators that create and destroy positive-energy (anti)particles, respectively.
% For our complex scalar field, these are two distinct particles, while for a real scalar field we say the particle is its own antiparticle.

\section{Interactions}
\label{app:01_qft_quantization_interactions}

\subsection{The interaction picture and Dyson's formula}
\label{sec:01_qft_quantization_interactions}

For treating interactions that are small perturbations to the free theory, it is most useful to employ the interaction picture of QM, a hybrid of the Schr\"odinger and Heisenberg pictures.
Recall that in the Schr\"odinger picture, operators are fixed while states evolve with time, and vice versa in the Heisenberg picture.
In the interaction picture, we split the Hamiltonian into the free ($H_0$) and interaction terms ($H_{\mathrm{int}}$), defining operators to evolve with the former and states with the latter.

The upshot of this in QFT is that the S-matrix element can be written according to \textit{Dyson's formula}:
\begin{equation}
	\label{eq:01_qft_interactions_dysons_formula}
	\braket{f|S|i} = \braket{f|T\exp\left(-i\int_{-\infty}^\infty H_I(t)dt\right)|i},
\end{equation}
where $T$ is the same time-ordering operator from Section~\ref{sec:01_qft_quantization_propagators} and $H_I$ is the time-evolved interaction Hamiltonian in the interaction picture:
\begin{equation}
	\label{eq:01_qft_interactions_interaction_hamiltonian}
	H_I(t) = e^{iH_0t}H_{\mathrm{int}}e^{-iH_0t}.
\end{equation}

Assuming a small $H_{\mathrm{int}}$, Dyson's formula can be Taylor expanded as:
\begin{multline}
	\label{eq:01_qft_interactions_dysons_expansion}
	\braket{f|S|i} = \braket{f|\identity|i} + (-i) \int_{-\infty}^\infty \braket{f|H_I(t)|i}dt \\
	+ \frac{(-i)^2}{2} \int\int_{-\infty}^{\infty} \braket{f|T H_I(t_1)H_I(t_2)|i}dt_1dt_2 + \ldots.
\end{multline}
The first term in the expansion is the free field term, which we ignore.\footnote{Often we simply define the ``interesting'' part as $\braket{f|S-\identity|i} \equiv iT$ and focus on calculating $T$.}
The $n$th term after that is of order $g^n$, where $g$ is the coupling constant of the interaction term.
Thus, this offers a prescription for calculating the S-matrix element up to any fixed order in the interaction strength.

Note that $\ket{i}$ and $\ket{f}$ are particle momentum eigenstates of the free theory. 
We can justify this intuitively by thinking of them as the states long before and after the interaction, when the interaction term is negligible.
Formally, there is in fact a complicated formula relating the free and interacting eigenstates; however, the proportionality factors cancel rather beautifully in the S-matrix element, allowing us to focus on only ``connected'' and ``amputated'' Feynman diagrams between the free eigenstates, defined in the next section.
This is illustrated (literally) for the vacuum states in Peskin and Schroeder~\cite{Peskin:1995ev} Chapter 4, and justified more generally by the LSZ reduction formula.


\subsection{First-order examples and the matrix element \texorpdfstring{$\mathcal M$}{M}}
\label{sec:01_qft_quantization_firstorder}

Let us look at the $n = 1$ and $n = 2\,$ S-matrix element terms from Eq.~\ref{eq:01_qft_interactions_dysons_expansion} for our scalar Yukawa theory (Eq.~\ref{eq:01_qft_interactions_yukawa}):
\begin{equation}
	\label{eq:01_qft_interactions_yukawa_smatrix_terms}
	\begin{split}
		\braket{f|S|i}^{(1)} &= -i \int_{-\infty}^\infty \braket{f|H_I(t)|i}dt = -ig \int d^4x \braket{f|\phi(x)\psi^\dagger(x)\psi(x)|i}, \\[1em]
		\braket{f|S|i}^{(2)} &= \frac{(-ig)^2}{2} \int d^4x \int d^4y \braket{f|T \phi(x)\psi^\dagger(x)\psi(x)\phi(y)\psi^\dagger(y)\psi(y)|i}.
	% \braket{f|S|i}^{(1)} = -i \int_{-\infty}^\infty \braket{f|H_I(t)|i}dt = -ig \int d^4x \braket{f|\phi(x)\psi^\dagger(x)\psi(x)|i}.
	\end{split}
\end{equation}
For given initial and final $N$-particle momentum states, these can be calculated manually by plugging in the field expansions (Eq.~\ref{eq:01_qft_quantization_fsf_fields} and \ref{eq:01_qft_quantization_complex_fields}).

For example, the first-order term $\braket{f|S|i}^{(1)}$ is non-zero only for processes like:
\begin{itemize}
	\item Meson decay $\phi \rightarrow \psi^\dagger\psi$:\quad $\ket{i} = \sqrt{2\svecp E} a^\dagger_{\cvec{p}} \ket{0}$,\quad $\ket{f} = \sqrt{4 E_{\cvec{q}_1} E_{\cvec{q}_2}} b^\dagger_{\cvec{q}_1} c^\dagger_{\cvec{q}_2} \ket{0}$; and
	\item Nucleon-antinucleon annihilation $\psi^\dagger\psi \rightarrow \phi$:\quad $\ket{i} = \sqrt{4 E_{\cvec{q}_1} E_{\cvec{q}_2}} b^\dagger_{\cvec{q}_1} c^\dagger_{\cvec{q}_2} \ket{0}$,\quad $\ket{f} = \sqrt{2\svecp E} a^\dagger_{\cvec{p}} \ket{0}$.
	% \item Meson emission $\psi \rightarrow \phi\psi$:\quad $\ket{i} = \sqrt{2E_{\cvec{q}_1}} c^\dagger_{\cvec{q}_1} \ket{0}$,\quad $\ket{f} = \sqrt{4\svecp E E_{\cvec{q}_2}} a^\dagger_{\cvec{p}\,} c^\dagger_{\cvec{q}_2} \ket{0}$.
\end{itemize}
The amplitude for these can be calculated to be:
\begin{equation}
	\label{app:01_qft_interactions_yukawa_smatrix_1}
	\braket{f|S|i}^{(1)} = -ig (2\pi)^4 \delta^{(4)}(p - q_1 - q_2),
\end{equation}
% The delta function ensures momentum conservation, and is in fact a general feature of all S-matrix elements.
% It also tells us that this process can only occur for $m \geq 2M$.
% We typically define
% \begin{equation}
% 	\label{eq:01_qft_interactions_matrix_element}
% 	\braket{f|S - \identity|i} \equiv i (2\pi)^4 \delta^{(4)}(\Sigma\,p) \mathcal M,
% \end{equation}
% where $\mathcal M$ is called the \textit{matrix element} of the process, and is the nontrivial component we must compute.
with the simple matrix element $\mathcal M = -g$.
Generally, however, calculating $\mathcal M$ each time using the field expansions can be quite cumbersome.
This is especially true at higher orders, which require \textit{Wick's theorem}~\cite{Wick:1950ee} to treat time-ordered fields.
We can avoid this by using \textit{Feynman diagrams}, and their associated rules, which allow us to simply read off a matrix element from a drawing of the process.

\subsection{Feynman diagrams}
\label{app:01_qft_quantization_feynman}

The conventions for Feynman diagrams in this dissertation are as follows:
\begin{enumerate}
	\item Time and momentum always flow from left to right. 
	Thus, the left-most particles represent the initial, and the right-most the final states.
	Momentum arrows are shown in Figure~\ref{fig:01_qft_interactions_feynman_first_order} explicitly but need not be.
	\item Mesons are plotted as dotted and nucleons as solid lines.
	\item Nucleon lines have arrows representing \textit{particle-flow}.
	For external (i.e., initial or final state) nucleons they point in the direction of momentum for particles and opposite for antiparticles. 
	Again, in general, particles need not be explicitly labeled as in in Figure~\ref{fig:01_qft_interactions_feynman_first_order} since the linestyles and particle-flow arrows suffice.
\end{enumerate}

As discussed above, only \textit{connected} and \textit{amputated} diagrams contribute to the S-matrix element, and we will focus on these.
Connected means that every part of the diagrams is connected to at least one external line, and amputated means that there are no loops on external lines.
Examples of disconnected and un-amputated diagrams are shown in Figure~\ref{fig:01_qft_interactions_feynman_disconnected}.
Interestingly, disconnected and un-amputated diagrams contribute to the vacuum and one-particle states, respectively, differing in the interacting versus free theory.

\begin{figure}[ht]
	\centering
	\begin{tikzpicture}
		\begin{feynman}
			% disconnected t-channel diagram
			\vertex (a1) at (0,1.5);
			\vertex (a2) at (0,-1.5);
			\vertex (b1) at (2,1.5);
			\vertex (b2) at (2,-1.5);
			\vertex (c) at (1, 0);

			\diagram*{
				(a1) -- [plain] (c) -- [plain] (b1),
				(a2) -- [plain] (c) -- [plain] (b2),
			};

			\vertex (f1) at (2.25, 0);
			\vertex (fup) at (2.25, 0.75);
			\vertex (fdown) at (2.25, -0.75);

			\diagram*{
				(f1) -- [plain, half left] (fup),
				(f1) -- [plain, half right] (fup),
				(f1) -- [plain, half left] (fdown),
				(f1) -- [plain, half right] (fdown),
			};

		\end{feynman}
	\end{tikzpicture}
	\hspace{3cm}
	\begin{tikzpicture}
		\begin{feynman}
			% disconnected t-channel diagram
			\vertex (a1) at (0,1.5);
			\vertex (a2) at (0,-1.5);
			\vertex (b1) at (2,1.5);
			\vertex (b2) at (2,-1.5);
			\vertex (c) at (1, 0);

			\diagram*{
				(a1) -- [plain] (c) -- [plain] (b1),
				(a2) -- [plain] (c) -- [plain] (b2),
			};

			\vertex (loopv1) at (1.33, 0.5);  % Place the loop vertex near the upper leg
			\vertex (loopv2) at (1.67, 1);  % Place the loop vertex near the upper leg

			\diagram*{
				(loopv1) -- [plain, half right, min distance=5mm] (loopv2),
			};
		\end{feynman}
	\end{tikzpicture}
	\vspace{5mm}
	\caption{Examples of a disconnected (left) and an un-amputated (right) Feynman diagram.}
	\label{fig:01_qft_interactions_feynman_disconnected}
\end{figure}



\subsubsection{Example: nucleon scattering}

Nucleon-nucleon scattering is the process: $\psi\psi \rightarrow \psi\psi$.
The lowest order at which this can occur is of $\mathcal O(g^2)$, as it requires at least two interaction vertices.
The possible second-order diagrams are shown in Figure~\ref{fig:01_qft_interactions_feynman_nn_scattering}.
We interpret them as nucleons interacting via the exchange of a meson.
As the nucleons are identical, we require two diagrams, for the two permutations of the two final states.

\begin{figure}[ht]
	\centering
	\captionsetup{justification=centering}
	\begin{tikzpicture}
		\begin{feynman}
			\vertex (a);
			\vertex [below=of a] (b);
			\vertex [above left=of a] (i1);
			\vertex [below left=of b] (i2);
			\vertex [above right=of a] (f1);
			\vertex [below right=of b] (f2);
			\diagram* {
				(a) -- [scalar, edge label={\footnotesize$k$}] (b),
				(i1) -- [fermion, edge label'={\footnotesize$q_{i1}$}] (a),
				(i2) -- [fermion, edge label={\footnotesize$q_{i2}$}] (b),
				(a) -- [fermion, edge label'={\footnotesize$q_{f1}$}] (f1),
				(b) -- [fermion, edge label={\footnotesize$q_{f2}$}] (f2),
			};
		\end{feynman}
	\end{tikzpicture}
	\hspace{3cm}
	\begin{tikzpicture}
		\begin{feynman}
			\vertex (a);
			\vertex [below=of a] (b);
			\vertex [above left=of a] (i1);
			\vertex [below left=of b] (i2);
			\vertex [above right=of a] (f1);
			\vertex [below right=of b] (f2);
			\diagram* {
				(a) -- [scalar, edge label={\footnotesize$k$}] (b),
				(i1) -- [fermion, edge label'={\footnotesize$q_{i1}$}] (a),
				(i2) -- [fermion, edge label={\footnotesize$q_{i2}$}] (b),
				(a) -- [fermion, edge label'={\footnotesize$q_{f2}$}] (f1),
				(b) -- [fermion, edge label={\footnotesize$q_{f1}$}] (f2),
			};
		\end{feynman}
	\end{tikzpicture}
	\vspace{5mm}
	\caption{The two lowest order nucleon scattering diagrams.}
	\label{fig:01_qft_interactions_feynman_nn_scattering}
\end{figure}

Using the first two Feynman rules, we find
\begin{equation}
	\label{eq:01_qft_interactions_nn_scattering_1}
	i \mathcal M = (-ig)^2 \cdot  \frac{1}{k^2 - m^2 + i\varepsilon}
\end{equation}
for both diagrams.
What remains is to enforce momentum conservation at each vertex.
For the left-most diagram, we see $k = q_{f1} - q_{i1} = q_{f2} - q_{i2}$, while for the right-most $k = q_{f2} - q_{i1} = q_{f1} - q_{i2}$.
Thus, the total matrix element is
\begin{equation}
	\label{eq:01_qft_interactions_nn_scattering_2}
	i \mathcal M = i(\mathcal M_{\mathrm{left}} + \mathcal M_{\mathrm{right}}) = (-ig)^2 \bigg[ \frac{1}{(q_{f1} - q_{i1})^2 - m^2} + \frac{1}{(q_{f2} - q_{i1})^2 - m^2} \bigg],
	% \begin{split}
	% 	i \mathcal M_1 = (-ig)^2 \cdot  \frac{1}{(q_{f1} - q_{i1})^2 - m^2 + i\varepsilon}, \\
	% 	i \mathcal M_2 = (-ig)^2 \cdot  \frac{1}{(q_{f2} - q_{i1})^2 - m^2 + i\varepsilon},
	% \end{split}
\end{equation}
where we have left out the $i\varepsilon$ term as there is no integral to perform.

Generally, we have to be careful with the relative signs of the matrix elements of different diagrams, corresponding to either constructive or destructive interference.
(In fact, Peskin and Schroeder list ``Figure out the overall sign of the diagram'' as a Feynman rule.)
In this case, we can reason physically that since nucleons are bosons, the amplitude will be symmetric under interchange of the two final states, and hence the two diagrams should be summed.

\subsubsection{Mandelstam variables}

To build some intuition for Mandelstam variables, let us sit in the center of mass (COM) frame, and define our coordinate frame such that incoming particles collide along the $z$-axis and scatter in the $y$-$z$ plane:
\begin{equation}
	\label{eq:01_qft_interactions_mandelstam_pcom}
	\begin{split}
		p_{i1} = (E, 0, 0, p) &\qquad p_{i2} = (E, 0, 0, -p) \\
		p_{f1} = (E, 0, p\sin\theta, p\cos\theta) &\qquad p_{f2} = (E, 0, -p\sin\theta, -p\cos\theta).
	\end{split}
\end{equation}
Then,
\begin{equation}
	\label{eq:01_qft_interactions_mandelstam_com}
	s = 4E^2, \quad t = -2p^2(1-\cos\theta), \quad u = -2p^2(1+\cos\theta).
\end{equation}
Thus, $s$ is the total energy in the COM frame squared --- hence, we usually refer to the COM energy as $\sqrt{s}$ --- while $t$ and $u$ are a measure of how much momentum is exchanged between the scattered particles.
For example, if $\theta = 0$, both particles continue in the same direction and $t = 0$, while if $\theta = \pi$, they completely 
reverse direction and the momentum transfer along the collision axis is maximized at $\sqrt{\abs{t}} = 2p$.



\section{Spinor field theory}
\label{sec:01_qft_spinors}

\begin{center}
	\centering
	\noindent
	\textit{...anything that comes back to itself with a minus sign after a 2$\pi$ rotation is always going to be a little strange.} --- David Tong~\cite{TongSM}
\end{center}

So far, we have focused on scalar fields, which live in the trivial representation of the Lorentz group and correspond to spin-$0$ bosons.
In this section, we discuss the field theory for spin-$\frac{1}{2}$ particles, or fermions, which constitute all matter in the universe.
% As discussed in Chapter~\ref{sec:01_symmetries_poincare}, all known elementary fermions are associated with \textit{Dirac spinor} fields, which transform under the $(\cnicefrac{1}{2},0) \oplus (0,\cnicefrac{1}{2})$ representation of the Lorentz group.
% We describe the EOM governing free spinor fields, known as the Dirac equation, in Section~\ref{sec:01_qft_spinors_dirac}.
% \TODO{We then...}
% We then quantize the free spinor field in Section~\ref{sec:01_qft_spinors_quantization} and finally discuss Feynman rules for an interacting spinor theory in Section~\ref{sec:01_qft_spinors_feynman}.

\subsection{The Dirac equation}
\label{sec:01_qft_spinors_dirac}

% \subsubsection{Historical development}

Like the Klein-Gordon equation, the Dirac equation was also an attempt at a relativistic version of the Schrödinger equation.
Before the development of QFT, the quantized KG equation was thought to produce negative probabilities due to its second derivative in time.\footnote{We now understand that the KG equation describes perfectly good scalar quantum fields, where the field-theoretic analog of the probability density is in fact the conserved charge of Eq.~\ref{eq:01_qft_quantization_complex_charge}, which is allowed to be negative.}
Dirac thus sought a relativistic \textit{first-order} differential equation in space and time.

Legend has it he was staring into a fire in Cambridge when he came up with an equation of the form
\begin{equation}
	\label{eq:01_qft_spinors_dirac}
	(i\gamma^\mu \partial_\mu - m)\psi = 0,
\end{equation}
where $\gamma^\mu$ are constants that will be defined in a moment, and $\psi$ is a complex field.
It is difficult to make this equation Lorentz covariant; indeed, it is impossible if $\psi$ is a scalar and each $\gamma^\mu$ is simply a number.\footnote{Or even two- or three-dimensional.}
Dirac's brilliant insight, however, was that it \textit{can} be covariant if $\gamma_\mu$ are $4\times 4$ complex matrices and $\psi$ a four component field.

The key is that $\gamma^\mu\partial_\mu$ is essentially the ``square-root'' of the d'Alembertian $\Box$ from the KG-equation:
\begin{equation}
	\label{eq:01_qft_spinors_dirac_wave}
	\gamma^\mu \partial_\mu \gamma^\nu \partial_\nu = \Box = \partial_\mu \partial^\mu,
\end{equation}
if (and only if) $\gamma^\mu$ and $\gamma^\nu$ satisfy the \textit{Clifford algebra}:
\begin{equation}
	\label{eq:01_qft_spinors_clifford_algebra}
	\{\gamma^\mu, \gamma^\nu\} = 2\eta^{\mu\nu},
\end{equation}
where $\{A, B\} = AB + BA$ is the anticommutator.
Dirac found this is possible with $4\times 4$ matrices such as
% Equation~\ref{eq:01_qft_spinors_clifford_algebra} defines \textit{Clifford algebra}, which has irreps only of dimension $4$, such as
\begin{equation}
	\label{eq:01_qft_spinors_gamma_matrices_weyl_basis}
	% \setlength{\arraycolsep}{8pt}
	\gamma^0 = \begin{pmatrix} 0 & \identity \\ \identity & 0 \end{pmatrix}, \quad 
	\gamma^i = \begin{pmatrix} 0 & \sigma^i \\ -\sigma^i & 0 \end{pmatrix},
\end{equation}
where $\sigma^i$ are the Pauli matrices (Chapter~\ref{sec:01_symmetries_so3}).
These are called the \textit{gamma}, or \textit{Dirac}, matrices, and plugging them into Eq.~\ref{eq:01_qft_spinors_dirac} yields the \textit{Dirac equation}, which can be written even more compactly by defining $\cslashed{\partial} \equiv \gamma^\mu \partial_\mu$:
\begin{equation}
	\label{eq:01_qft_spinors_dirac_slash}
	(i\cslashed{\partial} - m)\psi = 0.
\end{equation}

This equation is considered one of the most significant breakthroughs in theoretical physics, ``on par with the works of Newton, Maxwell, and Einstein before him''~\cite{hey2003new}.
The insights that followed, as we will outline in this section, provided a theoretical basis for fermion spin, implied the existence of antiparticles, and overall were foundational to the development of the SM.\footnote{These insights were so unexpected that Dirac thought ``his equation was more intelligent than its author''~\cite{brown1983birth}.}

\subsection{Spinors}
\label{sec:01_qft_spinors_spinors}

Before discussing solutions and quantization of the Dirac equation, let us examine what kind of object $\psi$ is.
A related property of the Clifford algebra is that
\begin{equation}
	\label{eq:01_qft_spinors_gamma_lorentz_generators}
	\Sigma_{\mu\nu} \equiv \frac{i}{4}[\gamma^\mu, \gamma^\nu]
\end{equation}
satisfies the Lorentz algebra (Eq.~\ref{eq:01_poincare_algebra_mmunu}).
This means $\Sigma_{\mu\nu}$ are generators of Lorentz transformations
\begin{equation}
	\label{eq:01_qft_spinors_spinor_lorentz_transformation}
	S[\Lambda] = e^{\frac{1}{2}\omega^{\mu\nu}\Sigma_{\mu\nu}}, 
\end{equation}
where $\Lambda$ is a Lorentz transformation with parameters $\omega^{\mu\nu}$, and $S[\Lambda]$ is a particular 4D representation.

It can be shown\footnote{See e.g. Ref.~\cite{LiuRQFT} Lecture 14.} that the Dirac equation is only Lorentz covariant if the components of $\psi$, $\psi_\alpha$, transform under this exact representation:
\begin{equation}
	\label{eq:01_qft_spinors_spinor_transformation}
	\psi_\alpha \rightarrow \psi'_\alpha = S[\Lambda]^\beta_{\ \alpha} \psi_\beta.
\end{equation}
It is important to note here that $S[\Lambda]$ is acting on the $\psi$ components --- also called the spinor indices -- and not on the spacetime coordinates $x^\mu$, which transform under the vector representation (Eq.~\ref{eq:01_lorentz_generators}).
Explicitly, including the spacetime coordinates, $\psi(x)$ transforms as:
\begin{equation}
	\label{eq:01_qft_spinors_spinor_transformation_x}
	\psi_\alpha(x) \rightarrow \psi'_\alpha(x') = S[\Lambda]^\beta_{\ \alpha} \psi_\beta(\Lambda^{-1}x),
\end{equation}
where both $S[\Lambda]$ and $\Lambda$ share the same transformation parameters $\omega^{\mu\nu}$ and thus correspond to the same Lorentz transformation.\footnote{$x' = \Lambda^{-1}x$ as this is an \textit{active} transformation, in which the field is shifted.}

\subsubsection{Dirac and Weyl spinors}

What is this representation?
Let's look at the rotation and boost generators individually:
\begin{equation}
	\label{eq:01_qft_spinors_spinor_generators}
	\Sigma_{0i} = \frac{i}{2} \begin{pmatrix} -\sigma^i & 0 \\ 0 & \sigma^i \end{pmatrix}, \quad
	\Sigma_{ij} = \frac{1}{2} \epsilon_{ijk} \begin{pmatrix} \sigma^k & 0 \\ 0 & \sigma^k \end{pmatrix}.
\end{equation}
Comparing this with Eqs.~\ref{eq:01_lorentz_irreps_weyl_left} and~\ref{eq:01_lorentz_irreps_weyl_right}, we see that the top left and bottom right blocks are exactly the left- and right-handed Weyl spinor irreps of the generators.
The handedness of a spinor is called its \textit{chirality}, and its physical significance will be discussed in a moment.
Thus, we identify $S[\Lambda]$ with the $(\cnicefrac{1}{2},0) \oplus (0,\cnicefrac{1}{2})$, or Dirac spinor, representation.

This also means that, in this basis of the gamma matrices (called the \textit{Weyl}, or \textit{chiral}, basis), the Dirac spinor $\psi$ can be decomposed into two Weyl spinors:
\begin{equation}
	\label{eq:01_qft_spinors_spinor_decomposition}
	\psi = \begin{pmatrix} \psi_L \\ \psi_R \end{pmatrix},
\end{equation}
which transform under their respective representations.
The two components can be isolated if we consider a fifth gamma matrix:
\begin{equation}
	\label{eq:01_qft_spinors_gamma_five}
	\gamma^5 = i\gamma^0\gamma^1\gamma^2\gamma^3 = \begin{pmatrix} -\identity & 0 \\ 0 & \identity \end{pmatrix}.
\end{equation}
$\gamma^5$ is similar to our main four matrices in that $\{\gamma^5, \gamma^\mu\} = 0$ and $(\gamma^5)^2 = \identity$.
Importantly, we see from its form in the Chiral basis that projection operators $P_L$ and $P_R$ can be defined as:
\begin{equation}
	\label{eq:01_qft_spinors_chiral_projection}
	P_L = \frac{1 - \gamma^5}{2}, \quad P_R = \frac{1 + \gamma^5}{2},
\end{equation}
which satisfy the projection property $P_{L/R}^2 = P_{L/R}$ and project out the left- and right-handed components of a Dirac spinor:
\begin{equation}
	\label{eq:01_qft_spinors_chiral_projection_action}
	P_{L} \begin{pmatrix} \psi_L \\ \psi_R \end{pmatrix} = \begin{pmatrix} \psi_L \\ 0 \end{pmatrix}, \quad P_{R} \begin{pmatrix} \psi_L \\ \psi_R \end{pmatrix} = \begin{pmatrix} 0 \\ \psi_R \end{pmatrix}.
\end{equation}
Note that while the specific form depends on the basis, the definitions in Eq.~\ref{eq:01_qft_spinors_chiral_projection} are basis-independent and can be considered to define chirality.

\subsubsection{Chirality}

The two Weyl spinor representations are related by a complex conjugation, meaning $\psi_L^*$ is a right-handed Weyl spinor, and vice versa.
For a complex scalar field, we interpreted the conjugate as the antiparticle.
The same interpretation applies here; hence, if a left-handed spinor describes a particle, its antiparticle is described by its conjugate, right-handed spinor.

The Dirac equation can be rewritten in the Weyl basis as two coupled equations of the Weyl spinors.
Let us define $\sigma^\mu = (\identity, \cvec{\sigma})$ and $\bar{\sigma}^\mu = (\identity, -\cvec{\sigma})$, so that
\begin{equation}
	\label{eq:01_qft_spinors_dirac_weyl}
	(i\gamma^\mu\partial_\mu - m)\psi = 
	\begin{pmatrix} 
		-m & i\sigma^\mu\partial_\mu \\ i\bar{\sigma}^\mu\partial_\mu & -m 
	\end{pmatrix}
	\begin{pmatrix} \psi_L \\ \psi_R \end{pmatrix} = 0.
\end{equation}
Hence, we see the mass term couples the left- and right-handed components. 
This is why all massive fermions must exist in pairs of particles and antiparticles.
An important special case, however, is for a neutral \textit{Majorana} fermion, where $\psi$ equals its charge conjugate $\psi^c$ (to be defined below). 
Such a particle is its own antiparticle and can have a left-handed- or right-handed-only mass term.
As discussed in Chapter~\ref{sec:01_symmetries_poincare}, the only Majorana candidate in the SM is the right-handed neutrino.

For $m = 0$, the Dirac equation decouples and leaves us with the \textit{Weyl equations} describing massless fermions:
\begin{equation}
	\label{eq:01_qft_spinors_weyl}
	i\sigma^\mu\partial_\mu \psi_R= 0, \quad i\bar{\sigma}^\mu\partial_\mu \psi_L = 0.
\end{equation}
In Fourier space, these are:
\begin{equation}
	\label{eq:01_qft_spinors_weyl_fourier}
	\begin{split}
		\sigma^\mu p_\mu \psi_R = (E - \cvec{\sigma}\cdot\cvec{p})\psi_R = 0 \quad \Rightarrow \quad  \frac{\cvec{\sigma}\cdot\cvec{p}}{\abs{\cvec{p}}}\, \psi_R = +\psi_R, \\
		\bar{\sigma}^\mu p_\mu \psi_L = (E + \cvec{\sigma}\cdot\cvec{p})\psi_L = 0 \quad \Rightarrow \quad \frac{\cvec{\sigma}\cdot\cvec{p}}{\abs{\cvec{p}}}\, \psi_L = -\psi_L,
	\end{split}
\end{equation}
where we used $E = \abs{\cvec{p}}$ for massless particles.
You may recall $\frac{\cvec{\sigma}\cdot\cvec{p}}{\abs{\cvec{p}}}$ is the helicity operator, projecting the particle spin along its momentum.
Thus, in the massless limit, we see that the left- and right-handed Weyl spinors are the $+1$ and $-1$ helicity eigenstates, respectively.

This is not the case for massive particles, as helicity is no longer Lorentz invariant: one can always boost into a frame where the momentum is inverted while the spin remains the same, changing the sign of the helicity.
Chirality is thus a more abstract concept for massive particles, related only to how they transform under Lorentz transformations.

Theories not symmetric under exchange of left- and right-handed components are called \textit{chiral}, and symmetric theories \textit{vector}.
QED and QCD are both vector theories, but weak interactions are, surprisingly, chiral.
This necessarily means it violates parity and charge conjugation symmetries ($P$ and $C$), which we will discuss soon in Section~\ref{sec:01_qft_spinors_cpt}.


\subsection{The Dirac Lagrangian}
\label{sec:01_qft_spinors_lagrangian}

Recall that to quantize the scalar theory, we first needed the Lagrangian and the classical solutions of the K-G equation, to then obtain Hamiltonian and canonical fields and Poisson brackets before finally promoting them to quantum commutatation relations.
We will proceed in similar (though condensed) fashion for the spinor theory, and first derive the Lagrangian corresponding to the Dirac equation.

Since we are no longer dealing with trivial representation of the Lorentz group, we have to be more careful with the types of terms we put into the Lagrangian; it must be composed of good Lorentz-invariant objects.
% The action is Lorentz invariant, so the Lagrangian must be composed of good Lorentz-covariant objects.
A first guess at a Lorentz scalar formed of spinors may be $\psi^\dagger\psi$.
This is indeed a scalar, but it is \textit{not} Lorentz invariant:
$\psi$ and $\psi^\dagger$ transform as $\psi\rightarrow S[\Lambda]\psi$, $\psi^\dagger\rightarrow \psi^\dagger S[\Lambda]^\dagger$ and, hence
\begin{equation}
	\label{eq:01_qft_spinors_lagrangian_scalar_wrong}
	\psi^\dagger\psi \rightarrow \psi^\dagger S[\Lambda]^\dagger S[\Lambda]\psi. % \neq \psi^\dagger\psi.
\end{equation}
However, recall from Chapter~\ref{sec:01_symmetries_poincare} that (finite-dimensional) representations of Lorentz transformations are not unitary.
(We can see this as well from the fact that the generators of $S[\Lambda]$ in Eq.~\ref{eq:01_qft_spinors_spinor_generators} are not anti-Hermitian.)
Thus, $S[\Lambda]^\dagger S[\Lambda] \neq 1$ in general and $\psi^\dagger\psi$ is not a Lorentz scalar.

Instead, with a bit of matrix algebra\footnote{See e.g. Schwartz~\cite{Schwartz:2014sze} Chapter 10.3}, one can show that
\begin{equation}
	\label{eq:01_qft_spinors_gamma0_inverse}
	\gamma^0 S[\Lambda] \gamma^0 = (S[\Lambda]^{-1})^\dagger,
\end{equation}
and hence
\begin{equation}
	\label{eq:01_qft_spinors_lagrangian_scalar}
	\psi^\dagger\gamma^0\psi \rightarrow \psi^\dagger S[\Lambda]^\dagger \gamma^0 S[\Lambda]\psi = \psi^\dagger\gamma^0 S[\Lambda]^{-1} S[\Lambda]\psi = \psi^\dagger\gamma^0\psi
\end{equation}
\textit{is} a Lorentz scalar.
Thus, we define $\bar\psi \equiv \psi^\dagger\gamma^0$ as the ``natural'' conjugate to $\psi$, and end up with a nice Lorentz scalar $\bar\psi \psi$ for our Lagrangian.

Similarly, one can show that $\bar\psi\gamma^\mu\psi$ transforms as a Lorentz $4$-vector and, hence, contracting it with $\partial_\mu$ as $\bar\psi\gamma^\mu\partial_\mu\psi$ yields another scalar.
These two terms, which are analogous to the mass and derivative terms a free complex scalar field (Eq.~\ref{eq:01_qft_symmetries_complex_lagrangian}), are enough to build the Dirac Lagrangian:
\begin{equation}
	\label{eq:01_qft_spinors_lagrangian}
	\mathcal{L} = i\bar\psi\gamma^\mu\partial_\mu\psi - m\bar\psi\psi = \bar\psi(i\cslashed{\partial} - m)\psi.
\end{equation}
One can check that the EL equations reproduce the Dirac equation for $\psi$ and $\bar\psi$.

\subsubsection{The U(1) conserved current}

As with the complex scalar field, observe that the Dirac Lagrangian is invariant under global \UU[1] symmetry $\psi \rightarrow e^{i\alpha}\psi$.
Using Noether's theorem, we can derive the conserved current and charge associated with this symmetry:
\begin{equation}
	\label{eq:01_qft_spinors_lagrangian_current}
	j^\mu = \bar\psi\gamma^\mu\psi, \quad Q = \int d^3x\, j^0 = \int d^3x\, \psi^\dagger\psi.
\end{equation}
As for the complex scalar field, these represent the electromagnetic $4$-current and charge, respectively --- a connection we will explore further in Section~\ref{sec:01_qft_gt_maxwell}.


\subsection{Quantizing the Dirac field}
\label{sec:01_qft_spinors_quantization}

\subsubsection{Solutions to the Dirac equation}

Before quantizing, we first need the classical solutions to the Dirac equation.
Multiplying both sides of it by $-(i\gamma^\mu\partial_\mu + m)$ gives us:
\begin{equation}
	\label{eq:01_qft_spinors_dirac_squared}
	-(i\gamma^\mu\partial_\mu + m)(i\gamma^\nu\partial_\nu - m)\psi = (\Box - m^2)\psi = 0,
\end{equation}
which means each component of $\psi$ individually satisfies the KG-equation.
Thus, we can assume similar plane wave solutions:
\begin{equation}
	\label{eq:01_qft_spinors_dirac_solution}
	\psi(x) = \int \frac{d^3p}{(2\pi)^3} \, u(p) e^{-ip\cdot x} + v(p) e^{ip\cdot x},
\end{equation}
where $u(p)$ and $v(p)$ are now spinors, and again we have positive and negative frequency solutions that correspond to particles and antiparticles, respectively, after quantization.

One can check using Fourier space, as we did for the Weyl equations, that
\begin{equation}
	\label{eq:01_qft_spinors_dirac_solution_up}
	u(p) = \begin{pmatrix} \sqrt{p \cdot \sigma}\, \xi \\ \sqrt{p \cdot \bar\sigma}\, \xi \end{pmatrix}, \quad
	v(p) = \begin{pmatrix} \sqrt{p \cdot \sigma}\, \eta \\ -\sqrt{p \cdot \bar\sigma}\, \eta \end{pmatrix}
\end{equation}
are general solutions to the Dirac equation, where $\xi$ and $\eta$ are the familiar two-component spinors from QM for spin-$\cnicefrac{1}{2}$ particles (although technically they do not have this interpretation before quantization).
As is conventional, we will use a basis of $\sigma_z$ eigenstates $\xi_1 = \eta_1 = (1, 0)^T$ and $\xi_2 = \eta_2 = (0, 1)^T$, corresponding to spin-up and spin-down, respectively.
% , with eigenvalues $+1$ and $-1$, respectively.

For example, in the rest frame $p_\mu = (m, 0, 0, 0)$, we have:
\begin{equation}
	\label{eq:01_qft_spinors_dirac_solution_rest}
	u(p)_1 = \sqrt{m} \begin{pmatrix} 1 \\ 0 \\ 1 \\ 0 \end{pmatrix}, \,
	u(p)_2 = \sqrt{m} \begin{pmatrix} 0 \\ 1 \\ 0 \\ 1 \end{pmatrix}, \,
	v(p)_1 = \sqrt{m} \begin{pmatrix} 1 \\ 0 \\ -1 \\ 0 \end{pmatrix}, \,
	v(p)_2 = \sqrt{m} \begin{pmatrix} 0 \\ 1 \\ 0 \\ -1 \end{pmatrix}.
\end{equation}
More generally, we can always orient a particle's 3-momentum along the $z$-axis, in which case:
\begin{equation}
	\label{eq:01_qft_spinors_dirac_solution_momentum}
\resizebox{\textwidth}{!}{$
	u(p)_1 = \begin{pmatrix} \sqrt{E - p_z} \\ 0 \\ \sqrt{E + p_z} \\ 0 \end{pmatrix}, \quad
	u(p)_2 = \begin{pmatrix} 0 \\ \sqrt{E - p_z} \\ 0 \\ \sqrt{E + p_z} \end{pmatrix} \quad
	v(p)_1 = \begin{pmatrix} \sqrt{E + p_z} \\ 0 \\ -\sqrt{E - p_z} \\ 0 \end{pmatrix}, \quad
	v(p)_2 = \begin{pmatrix} 0 \\ \sqrt{E + p_z} \\ 0 \\ -\sqrt{E - p_z} \end{pmatrix}.
$}
\end{equation}

\subsubsection{Quantization}

Now that we have a sensible Lagrangian and the classical solutions to the Dirac equation, the remaining steps to quantization follow closely that for our complex scalar field in Section~\ref{sec:01_qft_quantization_complex}, but with two notable differences.
The first is that we now must sum over the two spin components of $u_s(p)$ and $v_s(p)$, in addition to integrating over the momentum:
\begin{equation}
	\label{eq:01_qft_spinors_quantization}
	\begin{split}
		\psi(x) &= \sum_{s = 1, 2} \int \frac{d^3p}{(2\pi)^3} \left[\hat b^s_{\cvec{p}}\, u_s(p) e^{-ip\cdot x} + c^{s\dagger}_{\cvec{p}}\, v_s(p) e^{ip\cdot x}\right], \\
		\bar\psi(x) &= \sum_{s = 1, 2} \int \frac{d^3p}{(2\pi)^3} \left[\hat b^{s\dagger}_{\cvec{p}}\, \bar{u}_s(p) e^{ip\cdot x} + \hat c^s_{\cvec{p}}\,  \bar{v}_s(p) e^{-ip\cdot x}\right].
	\end{split}
\end{equation}
As before, we have positive and negative frequency solutions, with the $b/b^\dagger$ and $c/c^\dagger$ operators associated with particles of the same mass and opposite charge.

For spinors, we find that the $\hat b^{s\dagger} \ket{0}$ and $\hat c^{s\dagger} \ket{0}$ also have opposite spins, i.e. for the $z$-axis angular momentum operator $J_z$ (which can be derived through Noether's theorem as we did for the momentum operator in Section~\ref{sec:01_qft_classical_symmetries}):
\begin{equation}
	\label{eq:01_qft_spinors_spin_z}
	J_z\, \hat b^{s\dagger} \ket{0} = \pm\frac{1}{2} \hat b^{s\dagger} \ket{0}, \quad J_z\, \hat c^{s\dagger}\ket{0} = \mp \frac{1}{2} \hat c^{s\dagger}\ket{0}.
\end{equation}
By convention, we take $b^{s\dagger}$ and $b^s$ to be the creation and annihilation operators for the electron, and $c^{s\dagger}$ and $c^s$ for its antiparticle, the positron.
Thus, $\bar\psi_s(x)\ket{0}$ corresponds to an electron at $x$ with spin state $s$, and $\psi_s(x)\ket{0}$ to a positron at $x$ with the opposite spin state to $s$.

Through his equation, Dirac was the first to predict the existence of antimatter in 1930~\cite{Dirac:1930ek} (although he initially thought the electron's antiparticle was the proton).
This prediction was soon confirmed by the discovery of a particle with the same mass as the electron but opposite charge by Carl Anderson in a bubble chamber in 1932~\cite{Anderson:1932zz}.
Both were awarded the Nobel prize.

% made shocking prediction, and they were discovered.
% particles created by b have same mass as by a, but opposite charge and spin (?).

\subsubsection{The spin-statistics connection}

The second, extremely important difference from scalar quantization is that, because spinors are spin-$\frac{1}{2}$ particles, they must obey \textit{anticommutation relations}:
\begin{equation}
	\label{eq:01_qft_spinors_anticommutation}
	\begin{split}
		\{\psi_\alpha(x), \psi_\beta(y)\} =& \,\{\bar\psi_\alpha(x), \bar\psi_\beta(y)\} = 0, \\
		\{\psi_\alpha(x), \bar\psi_\beta(y)\} &= \delta_{\alpha\beta}\delta^3(\cvec{x} - \cvec{y}),
	\end{split}
\end{equation}
which also means the creation and annihilation operators satisfy:
\begin{equation}
	\label{eq:01_qft_spinors_anticommutation_operators}
	% \{a_s(p), a_{r}^\dagger(q)\} = \{b_s(p), b_{r}^\dagger(q)\} = (2\pi)^3\delta^3(\cvec{p} - \cvec{q})\delta_{sr}.
	\{\hat b^{s}_{\cvec{p}}, \hat b^{r\dagger}_{\cvec{q}}\} = \{\hat c^{s}_{\cvec{p}}, \hat c^{r\dagger}_{\cvec{q}}\} = (2\pi)^3\delta^3(\cvec{p} - \cvec{q})\delta_{sr}.
\end{equation}
Thus, unlike bosons, exchanging two particles yields a minus sign: $\hat b^{r\dagger}_{\cvec{p}_1} \hat b^{s\dagger}_{\cvec{p}_1} \ket{0} = -\hat b^{s\dagger}_{\cvec{p}_2} \hat b^{r\dagger}_{\cvec{p}_1} \ket{0}$, confirming that spinors obey Fermi-Dirac statistics and obey the Paul-Exclusion principle.

Were we to try and impose our earlier commutation relations for spinors (or indeed, any half-integer-spin field), we would run into several issues.
These include the time-ordered product in the $S$-matrix not being Lorentz invariant, and antiparticles contributing arbitrarily negative energies, making the theory unstable.
They are all related to the deep connection between spin and statistics: the requirement of Lorentz invariance, stability, and causality in a QFT necessitates that half-integer-spin particles obey Fermi-Dirac, and integer-spin particles Bose-Einstein statistics.\footnote{For more detailed discussion, see e.g. Peskin and Schroeder~\cite{Peskin:1995ev} Chapter 3.5 and Schwartz~\cite{Schwartz:2014sze} Chapter 12.}


\subsection{Interactions and Feynman rules}
\label{sec:01_qft_spinors_feynman}

Having quantized the free Dirac field, we now discuss interactions, again focusing on small (and renormalizable) perturbations to the free theory.
% \TODO{fix this...} As one may expect, the primary difference with respect to the complex scalar ``nucleon'' fields from Section~\ref{sec:01_qft_quantization_interactions} is 
We start by presenting the propagators for the Dirac field and then extending our scalar Yukawa theory from Section~\ref{sec:01_qft_interactions} to spinor ``nucleons''.

\subsubsection{Propagators}

We define the propagator for the Dirac field the same as for scalar fields in Section~\ref{sec:01_qft_quantization_propagators}:
% Recall from Section~\ref{sec:01_qft_quantization_propagators} that the propagator between spacetime points $x$ and $y$ is:
\begin{equation}
	\label{eq:01_qft_spinors_propagator}
	D_{\alpha\beta}(x - y) = \bra{0}\psi(x)_\alpha\bar\psi(y)_\beta\ket{0} = \int \frac{d^3p}{(2\pi)^3} \frac{1}{2E_p} \sum_s u^s_\alpha(p) \bar u^s_\beta(p) e^{-ip\cdot(x - y)},
\end{equation}
where $\alpha$ and $\beta$ index the spinor components.
Again, we have an extra sum over the spin states.
With some more matrix algebra one can show that these kinds of sums simplify nicely to
\begin{equation}
	\label{eq:01_qft_spinors_uvsum}
	\sum_s u^s_\alpha(p) \bar u^s_\beta(p) = (\cslashed{p} + m)_{\alpha\beta}, \quad \sum_s v^s_\alpha(p) \bar v^s_\beta(p) = (\cslashed{p} - m)_{\alpha\beta},
\end{equation}
so that we end up with, in momentum space, the Feynman propagator:
\begin{equation}
	\label{eq:01_qft_spinors_propagator_momentum}
	\Delta_F(p) \equiv \bra{0}T\psi(x)\bar\psi(y)\ket{0} =  \frac{i(\cslashed{p} + m)}{p^2 - m^2 + i\epsilon}.
\end{equation}
Note that we have now suppressed the spinor indices; $\Delta_F$ is still a $4\times4$ matrix in spinor space.
Note as well the relative minus sign in the time-ordering operator for fermions, due to exchanging the fields:
\begin{equation}
	\label{eq:01_qft_spinors_propagator_time_ordering}
	\bra{0}T\psi(x)\bar\psi(y)\ket{0} = \begin{cases} \bra{0}\psi(x)\bar\psi(y)\ket{0} & x^0 > y^0, \\ -\bra{0}\bar\psi(y)\psi(x)\ket{0} & x^0 < y^0. \end{cases}
\end{equation}

\subsubsection{External lines}

For scalars, external line terms such as $\phi \ket{p}$ simply contributed a factor of $1$ to the matrix element, where $\ket{p}$ is again a one-particle meson state with momentum $p$:
\begin{equation}
	\label{eq:01_qft_spinors_yukawa_scalar_external}
	\phi \ket{p} \sim \int \frac{d^3p'}{(2\pi)^3} \frac{1}{\sqrt{2E_{p'}}} a_{\cvec{p}'} e^{-ip'\cdot x} \sqrt{2E_{p}}\, a_{\cvec{p}}^\dagger \ket{0} = e^{-ip\cdot x} \ket{0}.
\end{equation}
(The $e^{-ip\cdot x}$ factor contributes only to the momentum conservation delta function in the $S$-matrix element.)
For spinors, we instead end up with a spinor factor.
For example, for an incoming fermion with momentum $q$ and spin $s$:
\begin{equation}
	\label{eq:01_qft_spinors_yukawa_spinor_external}
	\psi \ket{q, s} \sim \int \frac{d^3q'}{(2\pi)^3} \frac{1}{\sqrt{2E_{q'}}} \sum_s' b^{s'}_{\cvec{q}'} u^{s'}(q') e^{-iq'\cdot x} \sqrt{2E_{q}}\, b^{s\dagger}_{\cvec{q}} \ket{0} = u^s(q) e^{-iq\cdot x} \ket{0}.
\end{equation}
We can see looking at the form of the quantized fields (Eq.~\ref{eq:01_qft_spinors_quantization}), and which terms will contribute something non-zero, that incoming (outgoing) external fermions will be associated with a $u$ $(\bar u$) and antifermions with a $\bar v$ $(v$) factor.\footnote{The ``$\sim$'' becomes an ``$=$'' for a \textit{Wick contraction}, $\contraction{}{\phi}{}{\ket{p}} \phi \ket{p}$, which is what we deal with with time-ordered operator products.}

\subsubsection{Yukawa theory reloaded}

We now revisit Yukawa theory, the simplest possible theory of interactions for spinors.
The Lagrangian is the same as in Eq.~\ref{eq:01_qft_interactions_yukawa}, but now with $\psi$ a spinor:
\begin{equation}
	\label{eq:01_qft_spinors_yukawa_lagrangian}
	\mathcal{L} = \frac{1}{2}\partial^\mu\phi\partial_\mu\phi + i\bar\psi\cslashed{\partial}\psi - \frac{1}{2}m^2\phi^2 - M\bar\psi\psi - g\phi\bar\psi\psi.
\end{equation}
Note that through dimensional analysis, since $[M\bar\psi\psi] = [\bar\psi\cslashed{\partial}\psi] \mustequal 4$ we can deduce that $[\psi] = \frac{3}{2}$.
This means that (1) the Yukawa interaction is marginal, with $[\phi\bar\psi\psi] = 4$ and $[g] = 0$, and (2) importantly, there are no other renormalizable, Lorentz-invariant interactions we can write down for spinors with the fields at our disposal (modulo some $\gamma^5$'s thrown in, as we'll discuss in Section~\ref{sec:01_qft_spinors_cpt}).
Terms like $\psi\phi^2$, $\cslashed{\partial}\psi\phi$, or $\bar\psi\psi\phi^2$ are all either not Lorentz-scalars or of dimension $\geq 5$.
In this sense, because their possible interactions are so heavily constrained by their $\frac{3}{2}$-dimensionality, spinors in QFT are quite simple!
There is only one other spinor interaction in the SM, which we will see in Section~\ref{sec:01_qft_gt}, with gauge bosons.

We again refer to $\phi$ and $\psi$ as the ``meson'' and ``nucleon'' fields, which is slightly more accurate now since nucleons are in reality fermions.
The two main features missing from this theory are that the relevant mesons, the pions, are \textit{pseudoscalars} (to be discussed in the next section) and are a strong isospin triplet (to be described briefly in Chapter.~\ref{sec:01_sm_qcd}).
% However, \TODO{pseudo-scalar, isospin?}.

\begin{definition}
	\label{def:01_qft_spinors_yukawa_feynman}
	The Feynman rules in momentum space for spinor Yukawa theory are:
	\begin{enumerate}
	\item Vertices: \qquad
	\begin{tikzpicture}[baseline={([yshift=-0.8ex]current bounding box.center)}]
		\begin{feynman}[small]
			\vertex (a);
			\vertex [right=of a] (b);
			\vertex [above right=of b] (f1);
			\vertex [below right=of b] (f2);
			\diagram* {
				(a) -- [scalar] (b),
				(b) -- [fermion] (f1),
				(b) -- [anti fermion] (f2),
			};
		\end{feynman}
	\end{tikzpicture}
	$ = -ig$ \\[1em]
	\item Internal lines (propagators) \\[1em]
	\qquad\qquad Mesons: \quad
	\begin{tikzpicture}[baseline={([yshift=-1.8ex]current bounding box.center)}]
		\begin{feynman}[small]
			\vertex (a);
			\vertex [right=of a] (b);
			\diagram* {
				(a) -- [scalar, edge label={\footnotesize$p$}] (b) ,
			};
		\end{feynman}
	\end{tikzpicture}
	$\, = \cfrac{i}{p^2 - m^2 + i\varepsilon}$ \qquad
	Nucleons: \quad
	\begin{tikzpicture}[baseline={([yshift=-1.8ex]current bounding box.center)}]
		\begin{feynman}[small]
			\vertex (a);
			\vertex [right=of a] (b);
			\diagram* {
				(a) -- [fermion, edge label={\footnotesize$q$}] (b),
			};
		\end{feynman}
	\end{tikzpicture}
	$\, = \cfrac{i(\cslashed{q} + m)}{q^2 - M^2 + i\varepsilon}$ \\[1em]
	\item External lines (on-shell particles)  \\
    % \\[1.5em]
    % % \hspace*{-\leftmargini}
    % \begin{minipage}{0.99\textwidth}
    \begin{tabbing}
    Incoming mesons: 
    \hspace*{1.5cm}
    \=
	\begin{tikzpicture}[baseline={([yshift=-0.8ex]current bounding box.center)}]
		\begin{feynman}[small]
			\vertex (a);
			\vertex [right=of a] (b);
			\vertex [above right=of b] (f1);
			\vertex [below right=of b] (f2);
			\diagram* {
				(a) -- [scalar] (b),
				(b) -- [fermion] (f1),
				(b) -- [anti fermion] (f2),
			};
		\end{feynman}
	\end{tikzpicture}
    \hspace*{0.5cm}
    \=
    $ = 1$
    \\[1.5em]
    Outgoing mesons: 
    % \hspace*{1.5cm}
    \>
    \begin{tikzpicture}[baseline={([yshift=-0.8ex]current bounding box.center)}]
        \begin{feynman}[small]
            \vertex (a);
            \vertex [right=of a] (b);
            \vertex [above left=of a] (f1);
            \vertex [below left=of a] (f2);
            \diagram* {
                (f1) -- [fermion] (a),
                (f2) -- [anti fermion] (a),
                (a) -- [scalar] (b),
            };
        \end{feynman}
    \end{tikzpicture}
    % \hspace*{0.2cm}
    \>
    $ = 1$ \\[1.5em]
    Incoming nucleons: 
    % \hspace*{1.5cm}
    \>
    \begin{tikzpicture}[baseline={([yshift=-0.8ex]current bounding box.center)}]
        \begin{feynman}[small]
            \vertex (a);
            \vertex [right=of a] (b);
            \vertex [above right=of b] (f1);
            \vertex [below right=of b] (f2);
            \diagram* {
                (a) -- [fermion, momentum={\footnotesize$q, s$}] (b),
                (b) -- [fermion] (f1),
                (b) -- [scalar] (f2),
            };
        \end{feynman}
    \end{tikzpicture}
    % \hspace*{0.2cm}
    \>
    $ = u_s(q)$
    \\[1.5em]
    Outgoing nucleons: 
    \>
    % \hspace*{1.5cm}
    \begin{tikzpicture}[baseline={([yshift=-0.8ex]current bounding box.center)}]
        \begin{feynman}[small]
            \vertex (a);
            \vertex [right=of a] (b);
            \vertex [above left=of a] (f1);
            \vertex [below left=of a] (f2);
            \diagram* {
                (f1) -- [fermion] (a),
                (f2) -- [scalar] (a),
                (a) -- [fermion, momentum={\footnotesize$q, s$}] (b),
            };
        \end{feynman}
    \end{tikzpicture}
    \>
    $ = \bar{u}_s(q)$
    \\[1.5em]
    Incoming antinucleons: 
    \>
    \begin{tikzpicture}[baseline={([yshift=-0.8ex]current bounding box.center)}]
        \begin{feynman}[small]
            \vertex (a);
            \vertex [right=of a] (b);
            \vertex [above right=of b] (f1);
            \vertex [below right=of b] (f2);
            \diagram* {
                (a) -- [anti fermion, momentum={\footnotesize$q, s$}] (b),
                (b) -- [anti fermion] (f1),
                (b) -- [scalar] (f2),
            };
        \end{feynman}
    \end{tikzpicture}
    \>
    $ = \bar{v}_s(q)$
    \\[1.5em]
    Outgoing antinucleons: 
    \>
    \begin{tikzpicture}[baseline={([yshift=-0.8ex]current bounding box.center)}]
        \begin{feynman}[small]
            \vertex (a);
            \vertex [right=of a] (b);
            \vertex [above left=of a] (f1);
            \vertex [below left=of a] (f2);
            \diagram* {
                (f1) -- [anti fermion] (a),
                (f2) -- [scalar] (a),
                (a) -- [anti fermion, momentum={\footnotesize$q, s$}] (b),
            };
        \end{feynman}
    \end{tikzpicture}
    \>
    $ = v_s(q)$
    \end{tabbing}
	\item Impose momentum conservation at each vertex.
	\item Integrate over the momentum $k$ flowing through each loop.
	\item Figure out the sign based on statistics.
\end{enumerate}
\end{definition}

\subsubsection{Meson decay and the Higgs decay width}

\begin{figure}[ht]
	\centering
    % \captionsetup{justification=centering}
	\begin{tikzpicture}
		\begin{feynman}
			\vertex (a) {$\phi$};
			\vertex [right=of a] (b);
			\vertex [above right=of b] (f1) {$\bar u_{s_1}(q_1)$};
			\vertex [below right=of b] (f2) {$v_{s_2}(q_2)$};
			\diagram* {
				(a) -- [scalar] (b),
				(b) -- [fermion] (f1),
				(b) -- [anti fermion] (f2),
			};
		\end{feynman}
	\end{tikzpicture}
    \vspace{3mm}
    \caption{Tree-level Feynman diagram for meson decay via a Yukawa interaction.}
	\label{fig:01_qft_spinors_meson_decay}
\end{figure}


The matrix element for meson decay into a fermion-antifermion pair with spin and momentum $s_1, q_1$ and $s_2, q_2$, respectively, to first-order can be read off from the Feynman diagram in Figure~\ref{fig:01_qft_spinors_meson_decay}:
\begin{equation}
	\label{eq:01_qft_spinors_meson_decay_m}
	i \mathcal M = -ig\bar u_{s_1}(q_1) v_{s_2}(q_2)
\end{equation}

We can calculate the decay rate as in Section~\ref{sec:01_qft_interactions_decay}, except now we have to sum over the spins of the fermions:
\begin{equation}
	\label{eq:01_qft_spinors_meson_decay_rate}
	d\Gamma = \sum_{s_1, s_2}^2 \frac{1}{2m} \abs{\mathcal M}^2 d\Pi_{\mathrm{LIPS}} = \frac{g^2}{2m} \sum_{s_1, s_2}^2 \abs{\bar u_{s_1}(q_1) v_{s_2}(q_2)}^2 d\Pi_{\mathrm{LIPS}}.
\end{equation}
In the COM frame, we can choose $q_1 = (\frac{m}{2}, 0, 0, q)$ and $q_2 = (\frac{m}{2}, 0, 0, -q)$, with $q^2 = \frac{m^2}{4} - M^2$ by energy conservation.
Using the forms of $\bar u_s$ and $v_s$ we found in Eq.~\ref{eq:01_qft_spinors_dirac_solution_momentum}, we see that the sum over spin states simplifies nicely:
\begin{equation}
	\label{eq:01_qft_spinors_meson_decay_sum}
	\sum_{s_1, s_2}^2 \abs{\bar u_{s_1}(q_1) v_{s_2}(q_2)}^2 = 8q^2 = 2(m^2 - 4M^2).
\end{equation}
Since this is independent of the final state kinematics, the integral of $d\Pi_{\mathrm{LIPS}}$ is the same as for the scalar meson decay, and we obtain an the overall decay rate of:
\begin{equation}
	\label{eq:01_qft_spinors_meson_decay_rate_final}
	\Gamma = \frac{g^2m}{16\pi} \left(1 - \frac{4M^2}{m^2}\right)^{3/2}.
\end{equation}

As we hinted at in Section~\ref{sec:01_qft_interactions_decay}, this is in fact the decay width of the Higgs boson to fermions at tree level, if we plug in the Higgs Yukawa coupling constant $g_f = \cnicefrac{\sqrt{2} m_f}{v}$.
Here $m_f$ is the fermion mass and $v$ is the Higgs vacuum expectation value, $246\GeV$.
For example, for the $H\to \mu^+\mu^-$ decay, with $M = m_\mu = 105.7\MeV$ and $m = m_H = 125\GeV$, we get $\Gamma \approx 900\eV$, exactly in line with the predicted value~\cite{Denner:2011mq}!

One can similarly update our nucleon scattering amplitudes from Section~\ref{sec:01_qft_interactions_feynman}, which simply gain some inner products between the incoming and outgoing spin states (see e.g. Tong QFT~\cite{TongQFT} Chapter 5.7).
Notably, however, the $t$-channel and $u$-channel diagrams (Figure~\ref{fig:01_qft_interactions_feynman_nn_scattering}) now have a relative \textit{minus} sign, in accordance with Fermi-Dirac statistics.


\subsection{CPT Symmetries}
\label{sec:01_qft_spinors_cpt}

In this section, we discuss three important \textit{discrete} symmetries in QFT.
As discussed in Chapter~\ref{sec:01_symmetries_poincare}, the full Lorentz group includes the parity $P$ and time reversal $T$ operators.
In the $4$-vector representation, they have the simple forms $P = \diag(1, -1, -1, -1)$ and $T = \diag(-1, 1, 1, 1)$, meaning
\begin{equation}
	\label{eq:01_qft_spinors_pt4v}
	P: (t, \cvec{x}) \rightarrow (t, -\cvec{x}), \quad T: (t, \cvec{x}) \rightarrow (-t, \cvec{x}).
\end{equation}
However, their forms in other representations, such as spinors, are not as straightforward.

Observe also that all our complex Lagrangians so far have been invariant under some form of complex conjugation $\psi \leftrightarrow \psi^*$.
This represents another discrete symmetry, and since we know from Eq.~\ref{eq:01_qft_symmetries_u1_transformation} that complex conjugation inverts ``charge'', we call this charge conjugation, or $C$, symmetry.

All local, relativistic QFTs are necessarily invariant under the combined $CPT$ symmetry; this is known as the CPT theorem~\cite{Schwinger:1951xk, Luders:1954zz}.\footnote{One way to convince yourself of this is to check that all possible Lorentz scalar terms in the Lagrangian are invariant under $CPT$, as shown in Peskin and Shroeder~\cite{Peskin:1995ev} Chapter 3.6.}
Whether a theory is individually $C$, $P$, or $T$ invariant, however, must be determined by experiment,\footnote{And also somewhat by the requirement of anomaly cancellation; see e.g. Tong SM~\cite{TongSM} Chapter 4.} as we give examples of below.
If it is, we must impose the symmetries in our mathematical formulation by carefully defining the actions of the relevant operators; i.e., we have to consider how $\psi$ must transform under $P$ to maintain $P$-invariance of the Lagrangian, etc.

Such symmetries are crucial handles for understanding QFTs, particularly in the case of the weak and strong interactions for which we have otherwise little classical intuition.
By studying them, we often glean important insights into the theory, such as why certain processes are forbidden: for example, we now understand that the pion cannot decay into three photons because this would violate the $C$-invariance of QED.

\subsubsection{$P$- and $CP$-violation}

Historically, it was thought that parity individually is a universal symmetry of nature.
Indeed, this was verified experimentally for electromagnetism and the strong interaction, but, surprisingly, in 1956 an experiment measuring the isotropy of the beta decay of cobalt-60 to nickel-60 by Chien-Shiung Wu showed that the weak interaction in fact violates parity- (and $C$-) invariance~\cite{Wu:1957my}.
% Specifically, she measured that the electrons emitted in the decay were not emitted isotropically but were preferentially left-handed.
The two theorists, Yang Chen-Ning and Lee Tsung-Dao, who proposed this experiment won the Nobel prize the year after but, controversially, Wu did not.

It was then proposed by Lev Landau~\cite{Landau:1957tp} and others that perhaps the combined $CP$-symmetry
% (which implies $T$-symmetry, according to the CPT theorem)
is the true symmetry of nature.
As we define below, the $CP$ operation transforms a particle into its antiparticle, hence, $CP$-invariance can be thought of as saying the laws of physics are the same for particles and antiparticles.
This indeed appeared to be the case until 1964, when the Fitch-Cronin experiment discovered small, indirect $CP$-violation by the weak interaction by measuring decays of neutral kaons~\cite{Christenson:1964fg}, for which another Nobel prize was awarded to James Cronin and Val Fitch.
Since then, several experiments have observed both direct and indirect $CP$-violation, and quantifying the magnitude of $CP$-violation in different sectors of the SM remains an active area of research in HEP (see Ref.~\cite{ParticleDataGroup:2024cfk} Chapters 13-14 for a nice comprehensive review).

Interestingly, $CP$-violation is only possible through the weak interaction if there exist $\geq 3$ generations of fermions, whereas it is \textit{expected} for the strong interaction but not observed (the so-called ``strong $CP$ problem''~\cite{Wu:1991rw,Mannel:2007zz}.\footnote{The difference is a consequence of an ABJ anomaly for the \SU[2] gauge group (see e.g. Tong SM~\cite{TongSM} Chapter 5.1).}
Furthermore, the experimentally determined magnitude of $CP$-violation in the weak interaction is about $1000\times$ smaller than what is allowed~\cite{Mannel:2007zz, ParticleDataGroup:2024cfk}.
These mysterious ``coincidences'' --- Why did nature ``choose'' exactly the minimum number of generations needed for $CP$-violation? Why is there no strong $CP$-violation? etc. --- suggest deeper underlying physics, such as ``axions''~\cite{Dine:1981rt}.


% \subsubsection{Particles versus fields}

% Note that there is an ambiguity in our terminology related to whether these $C$, $P$, and $T$ operators are acting on particles or fields.
% For example, the $\hat C$ particle operator acting on a left-handed electron $\ket{p, s}_{e^-_L}$ is defined to transform it to a left-handed positron $\ket{p, s}_{e^-_L}$.
% However, the $C$ \textit{field} operator loosely transforms $\psi \rightarrow \bar\psi$, which is 
% gives a negative charged antiparticle state $\ket{p, s}$, but acting on the field $\psi$ it gives $\bar\psi$.


\subsubsection{Scalar fields}

We see from our complex scalar Lagrangian in Eq.~\ref{eq:01_qft_symmetries_complex_lagrangian} that it can only be invariant under $C$, $P$, or $T$ if they transform the field $\phi$ by at most a complex phase: $\phi \rightarrow e^{i\alpha}\phi$.
A further physical requirement, however, is that applying any of the operators twice should return the original field, which thus constrains the possible transformations to:
\begin{equation}
	\label{eq:01_qft_spinors_cpt_scalars}
	\begin{split}
		C\mathrm{:}\; \phi(t, \cvec{x}) &\rightarrow \pm\phi^*(t, \cvec{x}), \\
		P\mathrm{:}\; \phi(t, \cvec{x}) &\rightarrow \pm\phi(t, -\cvec{x}), \\
		T\mathrm{:}\; \phi(t, \cvec{x}) &\rightarrow \pm\phi(-t, \cvec{x}).
	\end{split}
\end{equation}
The time-reversal operation is a bit subtle, as it must be \textit{anti-unitary}.
We will not discuss it much further, although its implications can be fun to think about.
% The $\phi^*$ appears in the $T$ transformation because the time reversal operator $T$ is anti-unitary.

\subparagraph{Nomenclature} Whether a field transforms with a $+$ or $-$ sign under $P$ is called its \textit{intrinsic parity}, and similarly under $C$ its intrinsic $C$-parity.
We also refer to them as ``even'' or ``odd'' under the transformation, respectively.
In particular, an odd-parity scalar, i.e. one which transforms with a minus sign under parity, is called a \textit{pseudoscalar}.
The Higgs field, for example, is a scalar, while the pion is a pseudoscalar (as was determined based on nuclear interactions).

\subsubsection{Vector fields}

Though we introduce vector fields in detail in the next section, their transformation properties are analogous to scalars and simple enough to describe here:
\begin{equation}
	\label{eq:01_qft_spinors_cpt_vectors}
	\begin{split}
		C\mathrm{:}\; A^\mu(t, \cvec{x}) &\rightarrow \pm A^{\dagger\mu}(t, \cvec{x}), \\
		P\mathrm{:}\; A^\mu(t, \cvec{x}) &\rightarrow \pm \eta_{\mu\nu}A^\nu(t, -\cvec{x}), \\
		T\mathrm{:}\; A^\mu(t, \cvec{x}) &\rightarrow \mp \eta_{\mu\nu}A^\nu(-t, \cvec{x}),
	\end{split}
\end{equation}
where $\eta_{\mu\nu}$ is the Minkowski metric (i.e. $P$ and $T$ flip the sign of the first and the last three components of $A^\mu$, respectively).

We use similar ``odd'' and ``even'' nomenclature for vectors, with an odd-parity vector called a \textit{pseudovector}.
Recall for example that the electric and magnetic $3$-vector fields are vectors and pseudovectors, respectively.
Notably, the photon is odd under $C$ while the neutral pion; this explains why the pion can decay into two photons (since the two photons have a combined parity of $(-1)(-1) = +1$), but not to three, even though either would be allowed kinematically.


\subsubsection{Spinors: parity}

Spinors live in a more complicated representation of the Lorentz group, so it takes more work to derive their transformations.
On the other hand, this also means their properties and the physical consequences are more interesting.

% \subparagraph*{Parity}
If $P$ is a true symmetry of the theory, after a parity transformation $\psi'(x') = P\psi(x)P^\dagger$ must satisfy the parity-transformed Dirac equation:
\begin{equation}
	\label{eq:01_qft_spinors_cpt_parity}
	(i\gamma^\mu\partial'_\mu - m)\psi'(x') = 0,
\end{equation}
where $x^\mu \rightarrow x'^\mu = (x^0, -\cvec{x})$ and $\partial_\mu' \equiv \partial/\partial x'^\mu$ under parity.
One can see, by multiplying the original Dirac equation by $\gamma^0$, that this is satisfied if $\psi'(x') = \pm \gamma^0\psi(x)$:
\begin{equation}
	\label{eq:01_qft_spinors_cpt_parity2}
	\gamma^0(i\gamma^\mu\partial_\mu - m)\psi(x) = (i \gamma^\mu\partial'_\mu - m)\gamma^0\psi(x) = (i \gamma^\mu\partial'_\mu - m)\psi'(x') = 0.
\end{equation}
Again, the sign in the transformation indicates the intrinsic parity of the field.

Looking at the form of $\gamma^0$ and $\psi$ in the Weyl basis (Eqs.~\ref{eq:01_qft_spinors_gamma_matrices_weyl_basis} and~\ref{eq:01_qft_spinors_spinor_decomposition}), we see that the parity transformation swaps around left- and right-handed spinors:
\begin{equation}
	\label{eq:01_qft_spinors_cpt_parity3}
	P\psi_L(x)P^\dagger = \pm \psi_R(x'), \quad P\psi_R(x)P^\dagger = \pm \psi_L(x').
\end{equation}
Chirality being inverted makes sense given its (loose) connection to helicity, which is flipped under parity.
Similarly, remembering from Section~\ref{sec:01_qft_spinors_quantization} that particle and anti-particle solutions to the Dirac equation have the form $u(p) \propto (\xi, \xi)^T$ and $v(p) \propto (\eta, -\eta)^T$, respectively, we see that fermions and antifermions have even and odd parity, respectively.
The weak interaction breaks parity symmetry by interacting only with left-chiral fermions and right-chiral antifermions.

We can also check that the Lorentz scalars and vectors we constructed, $\bar\psi\psi$ and $\bar\psi\gamma^\mu\psi$, are indeed invariant under parity, e.g.:
\begin{equation}
	\label{eq:01_qft_spinors_cpt_scalar}
	P\mathrm{:}\; \bar\psi\psi \rightarrow \bar\psi'\psi' = \psi^\dagger\gamma^0\gamma^0\gamma^0\psi = \psi^\dagger\gamma^0\psi = \bar\psi\psi.
\end{equation}
However, we can also construct \textit{pseudo}scalars and \textit{pseudo}vectors by throwing in a $\gamma^5$ matrix: $\bar\psi\gamma^5\psi$ and $\bar\psi\gamma^5\gamma^\mu\psi$.
One can confirm this by grinding it out as above, or by simply looking at their form in the Weyl basis, e.g.:
\begin{equation}
	\label{eq:01_qft_spinors_cpt_pseudoscalar}
	\bar\psi\gamma^5\psi = \begin{pmatrix} \psi_L^\dagger & \psi_R^\dagger \end{pmatrix} \begin{pmatrix} 0 & \identity \\ \identity & 0 \end{pmatrix} \begin{pmatrix} -\identity & 0 \\ 0 & \identity \end{pmatrix} \begin{pmatrix} \psi_L \\ \psi_R \end{pmatrix} = \psi_L^\dagger\psi_R - \psi_R^\dagger\psi_L.
\end{equation}
We thus see that this will pick up an overall minus sign under $\psi_L \leftrightarrow \psi_R$.


\subsubsection{Spinors: charge conjugation and $CP$}

Under charge conjugation, $\psi \rightarrow \psi_c = C\psi^*$, where $C$ is a matrix that can mix up the spinor components.
We can follow similar reasoning as for parity to show that $\psi_c$ satisfies the Dirac equation only if:
\begin{equation}
	\label{eq:01_qft_spinors_cpt_charge_operator}
	C^{-1}\gamma^\mu C = -(\gamma^\mu)^*
\end{equation}
In the Weyl basis, this means $C = \pm i \gamma^2$ and thus
\begin{equation}
	\label{eq:01_qft_spinors_cpt_charge_conjugation}
	C\mathrm{:\,} \psi \rightarrow \psi_c = \pm i\gamma^2\psi^*,
\end{equation}
where as always the sign in the transformation indicates the intrinsic $C$-parity of the field.
Looking at the individual components:
\begin{equation}
	\label{eq:01_qft_spinors_cpt_charge_conjugation_weyl}
	C\mathrm{:\,} \psi_L \rightarrow \pm i\sigma^2\psi_R^*, \quad C\mathrm{:\,} \psi_R \rightarrow \mp i\sigma^2\psi_L^*.
\end{equation}
$\gamma^2$ and complex conjugation both flip chirality, so combined we see that charge conjugation retains it, transforming left-(right-)chiral fermions into left-(right-)chiral antifermions.
Thus, the weak interaction violates $C$-symmetry as well by coupling only to opposite-chirality fermions and antifermions.

Combining parity and charge conjugation gives us, in the Weyl basis:
\begin{equation}
	\label{eq:01_qft_spinors_cpt_cp}
	CP\mathrm{:\;} \psi \rightarrow \pm i \gamma^2\gamma^0\psi^*,
\end{equation}
or, in terms of the Weyl spinors:
\begin{equation}
	\label{eq:01_qft_spinors_cpt_cp_weyl}
	CP\mathrm{:\;} \psi_L \rightarrow \pm i\sigma^2\psi_L^*, \quad CP\mathrm{:\,} \psi_R \rightarrow \mp i\sigma^2\psi_R^*.
\end{equation}
The combination thus transforms fermions into their opposite-chirality antifermions, and vice versa.
Often, this transformation is considered to define the relation between particles and antiparticles, and is a better symmetry of the weak interaction (and, hence, the sm) than $C$ or $P$ individually.
However, as discussed above, it is violated as well, to a lesser extent, through the mixing of the three generations of fermions.


\subsubsection{Spinors: time reversal and CPT}

The time reversal operation is more subtle, as it is anti-unitary.
We will forego a detailed discussion of these subtleties (see e.g. Schwartz ~\cite{Schwartz:2014sze} Chapter 11.6), and note that the time reversal operator $T$ is defined to transform a Dirac spinor in the Weyl basis as:
\begin{equation}
	\label{eq:01_qft_spinors_cpt_time_reversal}
	T\mathrm{:\;} \psi(t, \cvec{x}) \rightarrow \pm i\gamma^1\gamma^3\psi(-t, \cvec{x}).
\end{equation}
It flips both the spin and momenta of the fermions, and is violated as well by the weak interaction (as it must be to ensure $CPT$-invariance, given $CP$-violation).

Finally, we can combine all these operations to obtain the $CPT$-transformation of the Dirac spinor:
\begin{equation}
	\label{eq:01_qft_spinors_cpt_cpt}
	CPT\mathrm{:\;} \psi(x) \rightarrow \pm -i\gamma^2\gamma^0\gamma^1\gamma^3\psi^*(-x) = -\gamma^5\psi^*(-x).
\end{equation}
This transforms a particle into an antiparticle reversed in space and time.

One interesting way of testing $CPT$-invariance is to measure the rates of a process' $CP$- and $T$-conjugates, and confirm that they are equal.
All experimental tests to this date have confirmed $CPT$-invariance~\cite{ParticleDataGroup:2024cfk}.


\section{Gauge theories}
\label{app:01_qft_gt}

\subsection{Why gauge invariance?}
\label{sec:01_qft_gt_why}

Gauge invariance is needed in order to embed massless spin-$1$ particles with only two physical DoFs (i.e., two polarizations), like the photon or gluons, into a spin-$1$ Lorentz tensor with 3 DoFs.\footnote{And similarly, for a massless spin-2 particle, i.e., the graviton.}
It also ensures the renormalizability of spin-$1$ fields (a Nobel-prize-winning result of `t Hooft in 1971~\cite{tHooft:1971akt, tHooft:1971qjg}).
The spin-$1$ tensor itself is simply an abstract mathematical convenience, which is redundant up to gauge transformations; only terms that are gauge invariant can be physical.

Why the charade of inventing fields with extra DoFs and then imposing an abstract symmetry to remove them?
The purely pragmatic answer is that it has proven the most expedient and precise way to calculate physical observables.
In this sense, it is not so dissimilar to using complex numbers to describe oscillating physical phenomena or renormalizing by imposing a cut-off and taking the limit as it goes to infinity.
They are all simply mathematical conveniences without necessarily any deeper physical meaning.

A less abstact alternative proposed in the 1960s, for example, was S-matrix theory which aimed to do away with all this QFT mumbo-jumbo and focus directly on the physical observables; however, to quote Weinberg, ``it got nowhere with real calculations''~\cite{WeinbergCERNLecture}.
On the other hand, despite its abstruseness, in the end with QFT we simply draw some pretty pictures and can quickly read off extremely sophisticated results (with some heavy caveats).
%  for weakly coupled theories and once we develop the mathematical apparatus like renormalization group theory and Fadeev-Popov ghosts).

A more poetic view is that, on top of their practicality, gauge symmetries offer a beautiful and elegant description of the fundamental forces of nature.
It is rather amazing that we need only to require a quantum \UU[1] gauge theory, with the usual physical properties of Lorentz invariance, causality, renormalizability etc., and QED naturally falls out!
To quote O'Raifeartaigh, ``gauge symmetry introduces all the physical radiation fields in a natural way and determines the form of their interactions, up to a few coupling constants. 
It is remarkable that this variety of physical fields, which play such different roles at the phenomenological level, are all manifestations of the same simple principle and even more remarkable that the way in which they interact with matter is prescribed in advance.''~\cite{ORaifeartaigh:1997dvq}.

This universality can be extended further with the geometric view of gauge theories, which has a strong connection to general relativity (GR).
Namely, invariance under gauge transformations in the SM is analogous to invariance under local diffeomorphisms (an \textit{external} local symmetry) in GR, and gauge fields are themselves connections on their respective gauge groups' fiber bundles, similar to the Levi-Civita connection between tangent bundles on a manifold.\footnote{See e.g. Frederic Schuller's lectures~\cite{SchullerGATP} for a great introduction to the geometric view of physics.}
Indeed, this is why the ``covariant derivative'' below is named so.

Finally, there is the possibility that gauge invariance is simply one of those mysteries of the SM, like flavor and charge quantization, which point to some deeper underlying physics we are yet to uncover.
For example, in string theory, gauge invariance can arise naturally in an EFT of massless spin-$1$ particles~\cite{Green:1987sp}.
Ultimately, these considerations are not particularly relevant to the experimental physics, but after all this is a dissertation for a doctorate of philosophy...
%  - These views have next to no practical consequence (yet?) but after all, this is a dissertation for a doctorate of philosophy. \\
