My interest in physics was ignited in the 10th grade by the discovery of the Higgs boson and Stephen Hawking's \textit{The Universe in a Nutshell}. 
These past five years, dedicated to understanding the Higgs and the mysteries of our early universe, have been a fulfillment of dreams then born and will be a part of my life I cherish forever.

I have many people to thank for allowing my childhood passion to flourish into this dissertation.
I start first and foremost with my parents and little Kli for their love and support at every step.
I thank as well my entire extended family for making me feel at home around the world, from Delhi to Pondicherry and California to Texas.
I am especially grateful to all my grandparents, whose wisdom, selflessness, and memory inspire me always.

I would not have survived this PhD --- particularly the two long years of lockdown and two long quarters of E\&M, without my amazing friends, old and new, in San Diego: Aneesh, Biswa, Chris, Davide, Dro, Elliot, Gerald, Hulk, and Varun (to name a few).
I thank as well my fellow CMS students, Farouk and Yanxi, with whom I moved across continents and conducted an ancillary PhD in ping pong.
Along with them, I thank my friends in Geneva and Chicago, including Fifi, Jay, Priyanka, and the LPC crew, for making my two years working at CERN and Fermilab so memorable.
Most of all, I thank Praniti, for her sweetness and support throughout.

Like the universe, my journey in high energy physics (HEP) began with a bang: the CERN Openlab summer student program.
It was a breathtaking experience, and I am grateful to Cliff, Frank, and my supervisor Maurizio for their support then and throughout my career since.
Maurizio, in particular, introduced me to the (Nobel-prize winning!) potential of AI in HEP, and I have been hooked ever since.

More importantly, he introduced me to Javier right as we were both, perhaps serendipitously, joining UCSD in the Fall of 2019.
Javier has been the most brilliant, kind, and supportive advisor I could have asked for, and I thank him for teaching me a lot more than just physics.
Along with him, I thank many awesome postdocs and scientists, including Cristina, Daniel, Nhan, Petar, and Si for their mentorship through the years.

Parts~\ref{part:sm}---\ref{part:aiml} are primarily original work for this dissertation, discussing the standard model, the CMS experiment and the LHC, AI and ML, and statistics, and building on several references listed therein.
On the topic of statistics, I thank Javier and Nick Smith for countless discussions (as well as their much-needed help with analysis software and the CMS combine tool)!

Part~\ref{part:ml4sim} presents novel methods for producing and validating fast simulations of the CMS detector using AI.
I thank Maurizio for introducing me to this topic as a summer student and his guidance since, and Javier for supporting this work, and all the research directions it bloomed, since the beginning of my PhD.
I also thank my fellow students on the topic, Mary and Breno; Nadya for her support during my time at CERN; Kevin Pedro for lending his expertise on CMS simulations; and the IRIS-HEP institute and the Fermilab LPC for supporting this work through the IRIS-HEP fellowship and the LPC AI fellowship and graduate scholarship, respectively.

Part~\ref{part:hh} describes searches for high energy Higgs-boson pair production in the \bbvv channel using data collected by the CMS experiment during Run 2 of the LHC.
I thank Cristina for her hands-on guidance on both the physics and technical aspects from the start and her patience as I refactored our codebase every week.
I thank as well Javier, Petar, Si, and the rest of our boosted double-Higgs working group, as well as Nhan and the DASZLE team, for their advice and support.
I thank finally Nick, Lindsey, and all the Coffea and Scikit-HEP developers for building a wonderful and supportive Pythonic HEP ecosystem.

Part~\ref{part:ml4jets} on the \jetnet library and Lorentz-equivariant ML represents a collection of work~\cite{Kansal:2023iqy, Tsan:2021brw, Hao:2022zns} on which I mentored some amazing students at UC San Diego and more.
I thank them all
for choosing me as their mentor, and Javier for encouraging and supporting us graduate students in engaging in so many rewarding mentorship opportunities. 

\

Chapter~\ref{sec:03_ml} is, in part, a reprint of the materials as they appear in
R. Kansal. ``Symmetry Group Equivariant Neural Networks,'' (2020);
and
NeurIPS, 2021, R. Kansal; J. Duarte; H. Su; B. Orzari; T. Tomei; M. Pierini; M. Touranakou; J.-R. Vlimant; and D. Gunopulos. Particle Cloud Generation with Message Passing Generative Adversarial Networks.
The dissertation author was the primary investigator and author of these papers.

Part~\ref{part:ml4sim} is, in part, a reprint of the materials as they appear in 
the NeurIPS ML4PS Workshop, 2020, R. Kansal; J. Duarte; B. Orzari; T. Tomei; M. Pierini; M. Touranakou; J.-R. Vlimant; and D. Gunopulos. Graph generative adversarial networks for sparse data generation in high energy physics;
NeurIPS, 2021, R. Kansal; J. Duarte; H. Su; B. Orzari; T. Tomei; M. Pierini; M. Touranakou; J.-R. Vlimant; and D. Gunopulos. Particle Cloud Generation with Message Passing Generative Adversarial Networks; and
Phys. Rev. D, 2023, R. Kansal; A. Li; J. Duarte; N. Chernyavskaya; M. Pierini; B. Orzari; and T. Tomei; Evaluating generative models in high energy physics; and
the NeurIPS ML4PS Workshop, 2024, A. Li; V. Krishnamohan; R. Kansal; J. Duarte; R. Sen; S. Tsan; and Z. Zhang; Induced generative adversarial particle transformers.
The dissertation author was the primary investigator and (co-)author of these papers.

Chapters~\ref{sec:05_smhh} and~\ref{sec:05_bsmxhy} and Part~\ref{part:hh}, in part, are currently being prepared for the publication of the material by the CMS collaboration.
The dissertation author was the primary investigator and author of these papers.

Part~\ref{part:ml4jets} is, in part, a reprint of the materials as they appear in
JOSS, 2023, R. Kansal; C. Pareja; Z. Hao; and J. Duarte; JetNet: A Python package for accessing open datasets and benchmarking machine learning methods in high energy physics; and
Eur. Phys. J. C, 2023, Z. Hao; R. Kansal; J. Duarte; and N. Chernyavskaya; Lorentz group equivariant autoencoders.
%  and
% the NeurIPS ML4PS Workshop, 2021, S. Tsan; R. Kansal; A. Aportela; D. Diaz; J. Duarte; S. Krishna; F. Mokhtar; J.-R. Vlimant; and M. Pierini; Particle graph autoencoders and differentiable, learned energy mover's distance;
The dissertation author was the primary investigator and (co-)author of these papers.