This dissertation describes efforts towards understanding the Higgs boson at the highest energies humanly accessible, using the CMS experiment at the Large Hadron Collider and advances in artificial intelligence (AI) and machine learning (ML).
We present searches for resonant and nonresonant Higgs-boson (\PH) pair production in the all-hadronic two beauty-quark and two vector boson (\PV) final state, using a novel strategy to measure the quartic \HHVV coupling and search for new Higgs-like bosons.
By targeting highly Lorentz-boosted Higgs pairs, we probe effects of potential new physics in the high energy Higgs sector, which could hold answers to fundamental mysteries of nature such as baryon asymmetry.

To enable these and future searches, we introduce as well significant developments in AI/ML, including in the identification of boosted \hvv decays with deep transformer networks and advances in AI-accelerated fast simulations of the CMS detector.
The latter notably includes the development of the first, highly performant generative models for point-cloud data in high energy physics, which have the potential to improve CMS' computational efficiency by up to three orders of magnitude.
We also highlight novel solutions to the important and challenging problems of calibrating and validating these ML techniques.
Finally, we present new approaches to search for new physics in a model-agnostic manner, using physics-informed ML methods equivariant to Lorentz transformations.

The quartic \HHVV coupling is observed (expected) to be constrained to $[-0.04, 2.05]$ ($[0.05, 1.98]$) at the 95\% confidence level relative to the standard model prediction, representing the second-most sensitive measurement of this coupling by CMS to date.
Exclusion limits on the production cross section of new heavy resonances decaying to two Higgs-like bosons are expected to be as low as 0.3\unit{fb} for high resonance masses.